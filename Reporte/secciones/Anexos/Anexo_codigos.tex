funcion\chapter{Scripts}

\section{Archivo \texttt{main.py}}
\begin{lstlisting}[
  caption={Implementación de \texttt{main.py}},
  style=pythonstyle,
  basicstyle=\ttfamily\footnotesize
]
import os
import pandas as pd
from AG_confs import *

from AG import genetic_algorithm
from libs.plot import *

def main():
    # Crear carpetas de salida generales
    os.makedirs("outputs", exist_ok=True)
    
    for func_key, func_data in FUNCTIONS.items():
        f_obj = func_data["func"]
        lb = func_data["lb"]
        ub = func_data["ub"]
        func_name = func_data["name"]
        num_runs = func_data["num_runs"]
        
        # Carpetas especificas de cada funcion
        func_folder = f"outputs/{func_key}"
        os.makedirs(func_folder, exist_ok=True)
        hist_folder = os.path.join(func_folder, "historiales")
        res_folder = os.path.join(func_folder, "resumenes")
        os.makedirs(hist_folder, exist_ok=True)
        os.makedirs(res_folder, exist_ok=True)
        
        print(f"\n==============================================")
        print(f"  FUNCIoN: {func_name}")
        print(f"==============================================")
        
        all_runs_history = []
        best_solutions_all_runs = []  # Guardaremos los mejores individuos (x1, x2) de cada corrida
        best_values_across_runs = []  # Guardaremos el best_val (fitness) de cada corrida
        
        for run in range(num_runs):
            print(f"\nEjecucion {run+1}/{num_runs}")
            
            (best_sol, best_val,
             worst_sol, worst_val,
             avg_sol,  avg_val,
             std_val,
             best_fitness_history,
             best_x1_history,
             best_x2_history,
             population_final,
             fitness_final,
             best_solutions_over_time) = genetic_algorithm(
                 f_obj, lb, ub,
                 pop_size=POP_SIZE,
                 num_generations=NUM_GENERATIONS,
                 tournament_size=TOURNAMENT_SIZE,
                 crossover_prob=CROSSOVER_PROB,
                 eta_c=ETA_C,
                 mutation_prob=MUTATION_PROB,
                 eta_mut=ETA_MUT
             )
            
            # 1) Guardar historial
            df_historial = pd.DataFrame({
                "Generacion": np.arange(1, NUM_GENERATIONS + 1),
                "Mejor x1": best_x1_history,
                "Mejor x2": best_x2_history,
                "Mejor Fitness": best_fitness_history
            })
            historial_filename = os.path.join(hist_folder, f"historial_run_{run+1}.csv")
            df_historial.to_csv(historial_filename, index=False)
            
            # 2) Guardar resumen de la corrida
            data_resumen = [
                ["Mejor", best_sol[0], best_sol[1], best_val],
                ["Media", avg_sol[0], avg_sol[1], avg_val],
                ["Peor", worst_sol[0], worst_sol[1], worst_val],
                ["Desv. estandar", np.nan, np.nan, std_val]
            ]
            df_resumen = pd.DataFrame(data_resumen, columns=["Indicador", "x1", "x2", "Fitness"])
            resumen_filename = os.path.join(res_folder, f"resumen_run_{run+1}.csv")
            df_resumen.to_csv(resumen_filename, index=False)
            
            print(df_resumen.to_string(index=False))
            
            all_runs_history.append(best_fitness_history)
            best_solutions_all_runs.append(best_sol)
            best_values_across_runs.append(best_val)
        
        # ===========================================
        #       RESUMEN GLOBAL DE LAS CORRIDAS
        # ===========================================
        best_values_arr = np.array(best_values_across_runs)
        solutions_arr = np.array(best_solutions_all_runs)  # Cada fila: [x1, x2] del mejor individuo de cada corrida
        
        # Para el "Mejor" y "Peor", buscamos el indice de la corrida con minimo y maximo fitness
        min_index = np.argmin(best_values_arr)
        max_index = np.argmax(best_values_arr)
        
        data_global = [
            ["Mejor (Fitness)", solutions_arr[min_index, 0], solutions_arr[min_index, 1], best_values_arr[min_index]],
            ["Peor (Fitness)", solutions_arr[max_index, 0], solutions_arr[max_index, 1], best_values_arr[max_index]],
            ["Media", np.mean(solutions_arr[:, 0]), np.mean(solutions_arr[:, 1]), np.mean(best_values_arr)],
            ["Desv. Estandar", np.std(solutions_arr[:, 0]), np.std(solutions_arr[:, 1]), np.std(best_values_arr)]
        ]
        df_global = pd.DataFrame(data_global, columns=["Indicador", "x1", "x2", "Fitness"])
        
        global_filename = os.path.join(res_folder, "resumen_global_corridas.csv")
        df_global.to_csv(global_filename, index=False)
        
        # (Opcional) Graficar evolucion del fitness de todas las corridas
        plot_evolucion_fitness(all_runs_history, func_key, func_name)
        
        # (Opcional) Graficar superficie 3D si la funcion es de 2 variables
        if len(lb) == 2:
            plot_surface_3d(f_obj, lb, ub, best_solutions_all_runs, func_key, func_name)

if __name__ == "__main__":
    main()

\end{lstlisting}

\section{Archivo \texttt{AG\_confs.py}}
\begin{lstlisting}[
  caption={Implementación de \texttt{AG\_confs.py}},
  style=pythonstyle,
  basicstyle=\ttfamily\footnotesize
]
import numpy as np
from libs.functions import langermann, drop_wave
# ---------------------------
# Parametros del algoritmo
# ---------------------------
POP_SIZE = 100            # Numero de individuos en la poblacion
NUM_GENERATIONS = 200     # Numero de generaciones
NUM_RUNS = 10             # Numero de ejecuciones completas (ciclos)

# Parametros de la funcion de Langermann
a = np.array([3, 5, 2, 1, 7])
b = np.array([5, 2, 1, 4, 9])
c = np.array([1, 2, 5, 2, 3])

# Parametros del torneo
TOURNAMENT_SIZE = 3  # Numero de individuos participantes en cada torneo

# Parametros del cruzamiento SBX
CROSSOVER_PROB = 0.9  # Probabilidad de aplicar cruzamiento
ETA_C = 15            # indice de distribucion para SBX

# Parametros de la mutacion polinomial
MUTATION_PROB = 10.0 / 2  # Probabilidad de mutar cada gen
ETA_MUT = 20              # indice de distribucion para mutacion polinomial

best_solutions_list = [] 
all_runs_history = []  # Para graficar luego

FUNCTIONS = {
    "langermann": {
        "func": langermann,
        "lb": np.array([0, 0]),
        "ub": np.array([10, 10]),
        "name": "Langermann",
        "num_runs": NUM_RUNS
    },
    "drop_wave": {
        "func": drop_wave,
        "lb": np.array([-5.12, -5.12]),
        "ub": np.array([5.12, 5.12]),
        "name": "Drop-Wave",
        "num_runs": NUM_RUNS
    }
}
\end{lstlisting}

\section{Archivo \texttt{AG.py}}
\begin{lstlisting}[
  caption={Implementación de \texttt{AG.py}},
  style=pythonstyle,
  basicstyle=\ttfamily\footnotesize
]
from AG_confs import *
from libs.selection import vectorized_tournament_selection
from libs.crossover import sbx_crossover_with_boundaries
from libs.mutation import polynomial_mutation_with_boundaries
from libs.auxiliaries_functions import initialize_population


# ---------------------------
# Funcion principal del GA
# ---------------------------
def genetic_algorithm(objective_func, lower_bound, upper_bound,
                      pop_size=POP_SIZE, num_generations=NUM_GENERATIONS,
                      tournament_size=TOURNAMENT_SIZE,
                      crossover_prob=CROSSOVER_PROB, eta_c=ETA_C,
                      mutation_prob=MUTATION_PROB, eta_mut=ETA_MUT):
    """
    Ejecuta el GA para la funcion objetivo dada y retorna:
      - best_solution, best_value
      - worst_solution, worst_value
      - avg_solution, avg_value
      - std_value (fitness)
      - best_fitness_history, best_x1_history, best_x2_history
      - population (final), fitness (final)
      - best_solutions_over_time (para animaciones)
    """
    num_variables = len(lower_bound)
    
    # 1) Inicializar poblacion
    population = initialize_population(pop_size, num_variables, lower_bound, upper_bound)
    fitness = np.array([objective_func(ind) for ind in population])
    
    best_fitness_history = []
    best_x1_history = []
    best_x2_history = []
    
    # Para animacion: almacenamos el mejor (x1, x2) en cada generacion
    best_solutions_over_time = np.zeros((num_generations, num_variables))
    
    for gen in range(num_generations):
        # Elitismo: guardar el mejor de la generacion actual
        best_index = np.argmin(fitness)
        best_fitness = fitness[best_index]
        elite = population[best_index].copy()
        
        best_fitness_history.append(best_fitness)
        best_x1_history.append(elite[0])
        best_x2_history.append(elite[1])
        best_solutions_over_time[gen, :] = elite
        
        new_population = []
        
        # Numero de padres necesarios (2 por cada par a generar)
        num_parents_needed = 2 * (pop_size - 1)
        winners, _ = vectorized_tournament_selection(fitness, num_parents_needed,
                                                     tournament_size, len(population),
                                                     unique_in_column=True, unique_in_row=False)
        
        # Generar un valor global para el crossover y otro para la mutacion (para toda la generacion)
        global_u = np.random.rand()
        global_r = np.random.rand()
        
        # Generar nueva poblacion
        for i in range(0, len(winners), 2):
            parent1 = population[winners[i]].copy()
            if i + 1 < len(winners):
                parent2 = population[winners[i+1]].copy()
            else:
                parent2 = parent1.copy()
            
            # Cruzamiento SBX usando el mismo u para todas las variables del cruce
            child1, child2 = sbx_crossover_with_boundaries(
                parent1, parent2, lower_bound, upper_bound,
                eta_c, crossover_prob, use_global_u=True, global_u=global_u
            )
            # Mutacion polinomial usando el mismo r para todas las variables del individuo
            child1 = polynomial_mutation_with_boundaries(
                child1, lower_bound, upper_bound,
                mutation_prob, eta_mut, use_global_r=True, global_r=global_r
            )
            child2 = polynomial_mutation_with_boundaries(
                child2, lower_bound, upper_bound,
                mutation_prob, eta_mut, use_global_r=True, global_r=global_r
            )
            
            new_population.append(child1)
            if len(new_population) < pop_size - 1:
                new_population.append(child2)
        
        # Convertir a array y evaluar el fitness de la nueva poblacion
        new_population = np.array(new_population)
        new_fitness = np.array([objective_func(ind) for ind in new_population])
        
        # Incorporar el individuo elite (elitismo)
        new_population = np.vstack([new_population, elite])
        new_fitness = np.append(new_fitness, best_fitness)
        
        # Actualizar la poblacion y su fitness para la siguiente generacion
        population = new_population.copy()
        fitness = new_fitness.copy()
    
    # Calcular estadisticas finales
    best_index = np.argmin(fitness)
    worst_index = np.argmax(fitness)
    best_solution = population[best_index]
    best_value = fitness[best_index]
    worst_solution = population[worst_index]
    worst_value = fitness[worst_index]
    avg_solution = np.mean(population, axis=0)
    avg_value = np.mean(fitness)
    std_value = np.std(fitness)
    
    return (best_solution, best_value,
            worst_solution, worst_value,
            avg_solution, avg_value,
            std_value,
            best_fitness_history,
            best_x1_history,
            best_x2_history,
            population,
            fitness,
            best_solutions_over_time)
\end{lstlisting}

\section{Archivo \texttt{selection.py}}
\begin{lstlisting}[
  caption={Implementación de \texttt{selection.py}},
  style=pythonstyle,
  basicstyle=\ttfamily\footnotesize
]
import numpy as np

def vectorized_tournament_selection(fitness, num_tournaments, tournament_size, pop_size,
                                    unique_in_column=True, unique_in_row=False):
    """
    Genera una matriz de torneos de forma vectorizada y retorna, para cada torneo,
    el indice del individuo ganador (el de menor fitness).
    
    Args:
      - fitness: array con los fitness de la poblacion (longitud = pop_size).
      - num_tournaments: numero de torneos a realizar (por ejemplo, el numero total
                         de selecciones de padres requeridas en la generacion).
      - tournament_size: numero de individuos que participan en cada torneo.
      - pop_size: tamano de la poblacion.
      - unique_in_column: si True, para cada posicion (columna) se eligen candidatos sin
                          repeticion entre torneos.
      - unique_in_row: si True, en cada torneo (fila) los candidatos seran unicos.
                    (Por defecto se permite repetir en la fila).
    
    Returns:
      - winners: array de indices ganadores (uno por torneo).
      - tournament_matrix: la matriz de candidatos (de tamano [num_tournaments x tournament_size]).
    """
    if unique_in_row:
        # Para cada torneo (fila), muestreamos sin reemplazo (cada fila es unica)
        tournament_matrix = np.array([np.random.choice(pop_size, size=tournament_size, replace=False)
                                      for _ in range(num_tournaments)])
    else:
        # Permitir repeticion en la fila, pero controlar la no repeticion en cada columna
        if unique_in_column:
            # Para cada columna, se genera una permutacion de los indices (o se usan numeros aleatorios sin repeticion)
            # Siempre que num_tournaments <= pop_size.
            if num_tournaments > pop_size:
                # Si se requieren mas torneos que individuos, se hace sin la restriccion por columna.
                tournament_matrix = np.random.randint(0, pop_size, size=(num_tournaments, tournament_size))
            else:
                cols = []
                for j in range(tournament_size):
                    # Para la columna j, se toman num_tournaments indices sin repeticion
                    perm = np.random.permutation(pop_size)
                    cols.append(perm[:num_tournaments])
                tournament_matrix = np.column_stack(cols)
        else:
            # Sin restricciones, se muestrea con reemplazo para cada candidato.
            tournament_matrix = np.random.randint(0, pop_size, size=(num_tournaments, tournament_size))
    
    # Para cada torneo (fila de la matriz), se selecciona el candidato con el menor fitness.
    winners = []
    for row in tournament_matrix:
        row_fitness = fitness[row]
        winner_index = row[np.argmin(row_fitness)]
        winners.append(winner_index)
    winners = np.array(winners)
    return winners, tournament_matrix
\end{lstlisting}

\section{Archivo \texttt{crossover.py}}
\begin{lstlisting}[
  caption={Implementación de \texttt{crossover.py}},
  style=pythonstyle,
  basicstyle=\ttfamily\footnotesize
]
import numpy as np

def sbx_crossover(parent1, parent2, lower_bound, upper_bound, eta, crossover_prob):
    """Realiza el cruzamiento SBX para dos padres y devuelve dos hijos."""
    child1 = np.empty_like(parent1)
    child2 = np.empty_like(parent2)
    
    if np.random.rand() <= crossover_prob:
        for i in range(len(parent1)):
            u = np.random.rand()
            if u <= 0.5:
                beta = (2*u)**(1/(eta+1))
            else:
                beta = (1/(2*(1-u)))**(1/(eta+1))
            
            # Genera los dos hijos
            child1[i] = 0.5*((1+beta)*parent1[i] + (1-beta)*parent2[i])
            child2[i] = 0.5*((1-beta)*parent1[i] + (1+beta)*parent2[i])
            
            # Asegurar que los hijos esten dentro de los limites
            child1[i] = np.clip(child1[i], lower_bound[i], upper_bound[i])
            child2[i] = np.clip(child2[i], lower_bound[i], upper_bound[i])
    else:
        child1 = parent1.copy()
        child2 = parent2.copy()
    
    return child1, child2

def sbx_crossover_with_boundaries(parent1, parent2, lower_bound, upper_bound,
                                  eta, crossover_prob, use_global_u=False, global_u=None):
    """
    Realiza el cruzamiento SBX con limites, usando formulas que ajustan beta en funcion
    de la cercania a las fronteras. Permite usar un unico 'u' global para todos los individuos 
    de la generacion o, de forma estandar, un 'u' distinto por cada gen.
    
    Args:
      - parent1, parent2: arrays con los padres.
      - lower_bound, upper_bound: arrays con los limites inferiores y superiores.
      - eta: indice de distribucion para SBX.
      - crossover_prob: probabilidad de aplicar el cruce.
      - use_global_u: si es True se utilizara el mismo valor de 'u' para todas las variables.
      - global_u: valor de 'u' que se aplicara globalmente (si se proporciona).
      
    Returns:
      - child1, child2: arrays con los hijos resultantes.
    """
    parent1 = np.asarray(parent1)
    parent2 = np.asarray(parent2)
    child1 = np.empty_like(parent1)
    child2 = np.empty_like(parent2)
    
    # Si no se realiza el crossover, retornamos copias de los padres.
    if np.random.rand() > crossover_prob:
        return parent1.copy(), parent2.copy()
    
    # Si se quiere usar un 'u' global y no se ha pasado, se genera uno.
    if use_global_u:
        if global_u is None:
            global_u = np.random.rand()
    
    for i in range(len(parent1)):
        x1 = parent1[i]
        x2 = parent2[i]
        lb = lower_bound[i]
        ub = upper_bound[i]
        
        # Aseguramos que x1 sea menor o igual que x2
        if x1 > x2:
            x1, x2 = x2, x1
        
        dist = x2 - x1
        if dist < 1e-14:
            child1[i] = x1
            child2[i] = x2
            continue
        
        # Calcular la minima distancia a las fronteras
        min_val = min(x1 - lb, ub - x2)
        if min_val < 0:
            min_val = 0
        
        beta = 1.0 + (2.0 * min_val / dist)
        alpha = 2.0 - beta**(-(eta+1))
        
        # Si se usa u global, se usa el mismo valor para cada variable
        if use_global_u:
            u = global_u
        else:
            u = np.random.rand()
        
        if u <= (1.0 / alpha):
            betaq = (alpha * u)**(1.0/(eta+1))
        else:
            betaq = (1.0 / (2.0 - alpha*u))**(1.0/(eta+1))
        
        # Calcular los hijos
        c1 = 0.5 * ((x1 + x2) - betaq * (x2 - x1))
        c2 = 0.5 * ((x1 + x2) + betaq * (x2 - x1))
        
        # Ajustar a los limites
        child1[i] = np.clip(c1, lb, ub)
        child2[i] = np.clip(c2, lb, ub)
    
    return child1, child2
\end{lstlisting}

\section{Archivo \texttt{mutation.py}}
\begin{lstlisting}[
  caption={Implementación de \texttt{mutation.py}},
  style=pythonstyle,
  basicstyle=\ttfamily\footnotesize
]
import numpy as np

def polynomial_mutation(child, lower_bound, upper_bound, mutation_prob, eta_mut):
    """Aplica mutacion polinomial a un hijo."""
    mutant = child.copy()
    for i in range(len(child)):
        if np.random.rand() < mutation_prob:
            r = np.random.rand()
            diff = upper_bound[i] - lower_bound[i]
            if r < 0.5:
                delta = (2*r)**(1/(eta_mut+1)) - 1
            else:
                delta = 1 - (2*(1-r))**(1/(eta_mut+1))
            mutant[i] = child[i] + delta * diff
            mutant[i] = np.clip(mutant[i], lower_bound[i], upper_bound[i])
    return mutant


def polynomial_mutation_with_boundaries(child, lower_bound, upper_bound,
                                        mutation_prob, eta_mut,
                                        use_global_r=False, global_r=None):
    """
    Aplica mutacion polinomial (con limites) a un vector 'child'.
    Puede usar un unico 'r' global para todas las variables (si use_global_r=True)
    o generar un 'r' distinto para cada variable.
    
    Args:
      - child : array-like
          Cromosoma (vector de decision) a mutar.
      - lower_bound, upper_bound : array-like
          Limites inferiores y superiores para cada variable.
      - mutation_prob : float
          Probabilidad de mutacion (en [0,1]) para cada variable.
      - eta_mut : float
          indice de distribucion para la mutacion.
      - use_global_r : bool
          Si True, se utiliza un unico valor 'r' para todas las variables.
      - global_r : float, opcional
          Valor de 'r' global a usar; si no se proporciona, se genera uno.
    
    Retutrns:
      - mutant : np.ndarray
          Nuevo vector mutado (manteniendo la dimension de 'child').
    """
    mutant = np.array(child, copy=True, dtype=float)
    num_vars = len(child)
    
    # Si se desea usar un 'r' global y no se ha proporcionado, se genera uno una sola vez.
    if use_global_r:
        if global_r is None:
            global_r = np.random.rand()
    
    for i in range(num_vars):
        # Decidir si mutar esta variable
        if np.random.rand() < mutation_prob:
            x = mutant[i]
            xl = lower_bound[i]
            xu = upper_bound[i]
            
            # Evitar division por cero si los limites son casi iguales
            if abs(xu - xl) < 1e-14:
                continue

            # d = distancia normalizada al limite mas cercano
            d = min(xu - x, x - xl) / (xu - xl)
            
            # Elegir r: global o individual para cada variable
            if use_global_r:
                r = global_r
            else:
                r = np.random.rand()

            nm = eta_mut + 1.0

            # Calcular delta_q segun el valor de r
            if r < 0.5:
                bl = 2.0 * r + (1.0 - 2.0 * r) * ((1.0 - d) ** nm)
                delta_q = (bl ** (1.0 / nm)) - 1.0
            else:
                bl = 2.0 * (1.0 - r) + 2.0 * (r - 0.5) * ((1.0 - d) ** nm)
                delta_q = 1.0 - (bl ** (1.0 / nm))
            
            # Calcular la nueva posicion y asegurarse que este dentro de los limites
            y = x + delta_q * (xu - xl)
            mutant[i] = np.clip(y, xl, xu)
    
    return mutant
\end{lstlisting}

\section{Archivo \texttt{auxiliares\_functions.py}}
\begin{lstlisting}[
  caption={Implementación de \texttt{auxiliares\_functions.py}},
  style=pythonstyle,
  basicstyle=\ttfamily\footnotesize
]
import numpy as np
# ---------------------------
# Funciones auxiliares del GA
# ---------------------------
def initialize_population(pop_size, num_variables, lower_bound, upper_bound):
    """Inicializa la poblacion uniformemente en el espacio de busqueda."""
    return np.random.uniform(low=lower_bound, high=upper_bound, size=(pop_size, num_variables))
\end{lstlisting}
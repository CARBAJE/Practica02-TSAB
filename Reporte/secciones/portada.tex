% portada
\begin{titlepage}
    \begin{center}
        \vspace*{1cm}

        \begin{tabular}{c@{\hspace{2cm}}c}
            \includegraphics[width=0.25\textwidth]{\logoInstitucion} &
            \includegraphics[width=0.4\textwidth]{\logoUniversidad}
        \end{tabular}

        \vspace{1cm}

        \textbf{\LARGE \nombreInstituto} \\
        \textbf{\Large \facultad} \\
        \vspace{0.3cm}
        \textbf{\large Materia: \materia} \\
        \textbf{\large Grupo: \grupo} \\
        \vspace{0.3cm}
        \textbf{\large Profesor: \profesora} \\
        \textbf{\large Periodo: \periodo} \\

        \vspace{0.5cm}

        \textbf{\LARGE Practica 02} \\
        \vspace{0.5cm}
        \textbf{\Large \textit{\textbf{Optimización de Carteras:} Un Enfoque de Algoritmo Genético con Manejo de Restricciones para la Selección de Activos}} \\

        \vspace{0.3cm}

        \textbf{\large Realizado por:} \\
        \textbf{\large \alumnoA\\ \alumnoB\\ \alumnoC\\ \alumnoD}

        % Bloque de Resumen y Abstract
        \begin{minipage}{0.8\textwidth}
            \textbf{Abstract:}\\[0.15cm]
            This paper presents an approach to solve the portfolio optimization problem using a genetic algorithm with constraint handling. The goal of portfolio optimization is to find the optimal allocation of capital among different financial assets to maximize return and/or minimize risk. The genetic algorithm is implemented to address this problem, incorporating constraint handling through the exterior penalty method to ensure that solutions satisfy real-world limitations, such as investment limits per asset and acceptable risk levels. The results obtained from applying the algorithm to a stock selection case study are presented and discussed.
        \end{minipage}

        \begin{minipage}{0.8\textwidth}
            \textbf{Resumen:}\\[0.15cm]
            Esta práctica aborda el problema de la optimización de carteras de inversión, donde el objetivo es determinar la distribución óptima del capital entre diferentes activos financieros para maximizar el retorno y/o minimizar el riesgo. Se implementa un algoritmo genético para resolver este problema, incorporando el manejo de restricciones mediante el método de penalización exterior para asegurar que las soluciones obtenidas cumplan con las limitaciones del mundo real, como límites de inversión por activo y niveles de riesgo aceptables. Se presentan y discuten los resultados obtenidos al aplicar el algoritmo a un caso de estudio de selección de acciones.
        \end{minipage}

        \vspace{0.3cm}

        \textbf{\large Fecha: \today}

    \end{center}
\end{titlepage}

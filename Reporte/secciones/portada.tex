% portada
\begin{titlepage}
    \begin{center}
        \vspace*{1cm}

        \begin{tabular}{c@{\hspace{2cm}}c}
            \includegraphics[width=0.25\textwidth]{\logoInstitucion} &
            \includegraphics[width=0.4\textwidth]{\logoUniversidad}
        \end{tabular}

        \vspace{1cm}

        \textbf{\LARGE \nombreInstituto} \\
        \textbf{\Large \facultad} \\
        \vspace{0.5cm}
        \textbf{\large Materia: \materia} \\
        \textbf{\large Grupo: \grupo} \\
        \vspace{0.5cm}
        \textbf{\large Profesor: \profesora} \\
        \textbf{\large Periodo: \periodo} \\

        \vspace{0.75cm}

        \textbf{\LARGE Practica 01} \\
        \vspace{0.5cm}
        \textbf{\Large \textit{Maximizar Contraste en Imagenes Medicas.}} \\

        \vspace{0.3cm}

        \textbf{\large Realizado por:} \\
        \textbf{\large \alumnoA \\ \alumnoB \\ \alumnoC \\ \alumnoD}

        % Bloque de Resumen y Abstract
        \begin{minipage}{0.8\textwidth}
            \textbf{Abstract:}\\[0.3cm]
            This report describes the design, implementation, and evaluation of a Genetic Algorithm (GA) aimed at minimizing multimodal functions. Two benchmark functions—Langermann and Drop-Wave—serve as test cases. The GA employs tournament selection, Simulated Binary Crossover (SBX) with boundary handling, and polynomial mutation with boundaries. In addition to detailing the algorithm’s components, the report outlines the experimental setup, visualization techniques, and potential avenues for future improvements.
        \end{minipage}

        \begin{minipage}{0.8\textwidth}
            \textbf{Resumen:}\\[0.3cm]
            En este reporte se describe el diseño, implementacion y evaluacion de un Algoritmo Genetico (GA) adecuado a minimizar funciones multi-modales. Con dos funciones de evaluación comparativa (Benchmark functions)--\textit{Langermann} y \textit{Drop-Wave}-- sirven como ejemplificacion de casos de uso. El GA implementa: \textit{Selección por Torneo}, \textit{Simulated Binary Crossover (SBX)} con manejo de limites, \textit{Mutación Polinomial} con uso de cotas y \textit{Sustitución Extinitiva con Elitismo}. Ademas de detallar los componentes de los algoritmos, el reporte añade los valores de los parametros utilizados, tecnicas de visualizacion y potenciales caminos para su mejora continua.
        \end{minipage}
        
        \vspace{0.3cm}

        \textbf{\large Fecha: \today}

    \end{center}
\end{titlepage}

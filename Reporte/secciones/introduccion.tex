\chapter{Introducción}
En el presente trabajo se implementa y analiza un algoritmo genético (AG) como herramienta de optimización para resolver un problema restringido proveniente del ámbito financiero. Los algoritmos genéticos, inspirados en los principios de la evolución biológica, constituyen una técnica robusta de búsqueda estocástica ampliamente utilizada en la optimización de funciones no lineales, especialmente en contextos donde las restricciones y la dimensionalidad del espacio de búsqueda dificultan la aplicación de métodos deterministas tradicionales.
La arquitectura del algoritmo propuesto incorpora operadores clásicos de selección por torneo, cruzamiento mediante Simulated Binary Crossover (SBX) con preservación de límites, y mutación polinomial, todos ellos diseñados para mantener la diversidad poblacional y facilitar la exploración eficiente del espacio de soluciones. Además, se introduce un enfoque de penalización basado en la agregación de violaciones de restricciones, ponderadas por un parámetro $\lambda$, que permite adaptar el esquema de evaluación de aptitud a entornos restringidos.
El sistema ha sido diseñado para ejecutarse múltiples veces sobre una o más funciones objetivo, con el fin de obtener estadísticas robustas sobre el desempeño del AG. Para cada corrida, se registra el historial evolutivo del mejor individuo por generación, el comportamiento de las violaciones de restricciones y los valores estadísticos agregados (mejor, peor, promedio y desviación estándar). Asimismo, se generan representaciones gráficas que permiten visualizar la evolución del fitness y la viabilidad de las soluciones, así como superficies tridimensionales del espacio de búsqueda para funciones con dos variables

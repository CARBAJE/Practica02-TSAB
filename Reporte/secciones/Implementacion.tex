\chapter{Implementaci\'on}

El proyecto se ha desarrollado siguiendo una arquitectura modular\footnote{Se puede encontrar el codigo perteneciente a cada funcion dentro de} que permite separar claramente las distintas funcionalidades y facilita tanto el mantenimiento como la ampliaci\'on futura. A continuaci\'on, se detallan los principales componentes y c\'omo se integran en el sistema:

\section{Funciones Objetivo}

\subsection{Descripci\'on}
Las funciones objetivo definen el problema a optimizar. En este proyecto se incluyen dos funciones:
\begin{itemize}
    \item \textbf{Langermann}: Una funci\'on multimodal que combina componentes cosenoidales y exponenciales, generando m\'ultiples \'optimos locales.
    \item \textbf{Drop-Wave}: Una funci\'on bidimensional con una superficie ondulada, ideal para evaluar el desempe\~no del algoritmo en entornos con picos y valles.
\end{itemize}

\subsection{Implementaci\'on}
Estas funciones se encuentran en el archivo \texttt{functions.py} y est\'an parametrizadas con sus respectivos l\'imites de b\'usqueda (por ejemplo, Langermann se define en el intervalo [0,10] para cada variable, mientras que Drop-Wave utiliza los l\'imites [-5.12, 5.12]).

\section{M\'odulos del Algoritmo Gen\'etico}

El n\'ucleo del algoritmo gen\'etico se distribuye en varios m\'odulos:

\subsection{Inicializaci\'on}
\textbf{Funci\'on}: \texttt{initialize\_population}  
\textbf{Ubicaci\'on}: \texttt{libs/auxiliaries\_functions.py}  
\textbf{Descripci\'on}: Genera la poblaci\'on inicial de manera uniforme en el espacio de b\'usqueda.

\subsection{Selecci\'on}
\textbf{Funci\'on}: \texttt{vectorized\_tournament\_selection}  
\textbf{Ubicaci\'on}: \texttt{libs/selection.py}  
\textbf{Descripci\'on}: Se usa un enfoque de torneos para la selecci\'on de padres, empleando operaciones vectorizadas con NumPy.

\subsection{Cruzamiento}
\textbf{Funciones}: \texttt{sbx\_crossover}, \texttt{sbx\_crossover\_with\_boundaries}  
\textbf{Ubicaci\'on}: \texttt{libs/crossover.py}  
\textbf{Descripci\'on}: Implementa el operador SBX (Simulated Binary Crossover) con y sin control de l\'imites.

\subsection{Mutaci\'on}
\textbf{Funciones}: \texttt{polynomial\_mutation}, \texttt{polynomial\_mutation\_with\_boundaries}  
\textbf{Ubicaci\'on}: \texttt{libs/mutation.py}  
\textbf{Descripci\'on}: Aplica mutaci\'on polinomial, con control opcional de l\'imites para mantener la viabilidad de las soluciones.

\subsection{Ejecuci\'on del Algoritmo}
\textbf{Funci\'on}: \texttt{genetic\_algorithm}  
\textbf{Ubicaci\'on}: \texttt{AG.py}  
\textbf{Descripci\'on}: Gestiona el ciclo evolutivo completo del algoritmo gen\'etico.

\section{Visualizaci\'on y Almacenamiento}

\subsection{Visualizaci\'on}
\textbf{M\'odulo}: \texttt{libs/plot.py}  
\textbf{Funciones}: \texttt{plot\_evolucion\_fitness}, \texttt{plot\_surface\_3d}  
\textbf{Descripci\'on}: Permite analizar la evoluci\'on del fitness y visualizar la superficie de las funciones objetivo.

\subsection{Almacenamiento}
\textbf{Estructura de Carpetas}:
\begin{itemize}
    \item Directorio \texttt{outputs} organizado en subcarpetas por funci\'on.
    \item Historiales en archivos CSV con datos de fitness y variables.
    \item Res\'umenes estad\'isticos de cada corrida.
\end{itemize}

\textbf{Integraci\'on}: \texttt{main\_script.py} ejecuta el algoritmo para cada funci\'on definida en \texttt{AG\_confs.py}.

\textbf{Escalabilidad}: La arquitectura modular permite agregar nuevas funciones objetivo y modificar operadores gen\'eticos sin afectar la estructura base.
\chapter{Conclusiones}
El proyecto demuestra la efectividad de los algoritmos genéticos en la optimización de funciones complejas. La implementación de técnicas avanzadas, como el cruzamiento SBX y la mutación polinomial, combinada con una estrategia de selección por torneo, ha permitido explorar de manera eficiente el espacio de soluciones y mejorar progresivamente el fitness de la población. Las herramientas de visualización y el almacenamiento de resultados facilitan el análisis del comportamiento del algoritmo y ofrecen una base sólida para futuras mejoras o aplicaciones a problemas más complejos.

Además, el algoritmo genético desarrollado constituye un marco sólido y escalable para la optimización de funciones complejas. Los resultados obtenidos respaldan la viabilidad del enfoque y abren la puerta a futuras investigaciones, ya sea para afinar los parámetros del algoritmo o para extender su aplicación a problemas con mayores dimensiones o características más complejas. Esta base permite, además, la incorporación de mejoras y la adaptación del método a diferentes contextos, consolidando su utilidad en el ámbito de la optimización computacional.
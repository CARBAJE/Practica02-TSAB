\documentclass{report}

\usepackage[utf8]{inputenc}
\usepackage[spanish]{babel}
\usepackage{geometry}
\usepackage{graphicx}
\usepackage{titlesec}
\usepackage{lipsum}
\usepackage[dvipsnames]{xcolor}
\usepackage[fleqn]{mathtools}
\usepackage{booktabs}
\usepackage{amsmath}
\usepackage{amssymb} % Para los símbolos de conjuntos de números
\usepackage{latexsym}
\usepackage{nccmath}
\usepackage{multicol}
\usepackage{listings}
\usepackage{tasks}
\usepackage{color}
\usepackage{float}
\usepackage{enumitem}
\usepackage{longtable}
\usepackage{makecell}
\usepackage{caption}
\usepackage[parfill]{parskip}
\usepackage{lipsum}
\usepackage{enumitem}
\usepackage[hidelinks]{hyperref}% Para manejar enlaces
\usepackage{fancyhdr}
\usepackage{chappg}
\usepackage{adjustbox}
\usepackage{tocloft}
\usepackage{soul}
\usepackage{tikz}
\usetikzlibrary{arrows}
\usepackage{pgfplots}
\pgfplotsset{compat=1.18}
%Definicion de colores
\definecolor{colorIPN}{rgb}{0.5, 0.0,0.13}
\definecolor{colorESCOM}{rgb}{0.0, 0.5,1.0}
\definecolor{myorange}{RGB}{255, 204, 153}
\graphicspath{ {imagenes/} }
%Configuraciones extras de colores
\definecolor{codegreen}{rgb}{0,0.6,0}
\definecolor{codered}{rgb}{0.6,0,0}
\definecolor{codegray}{rgb}{0.5,0.5,0.5}
\definecolor{codepurple}{rgb}{0.58,0,0.82}
\definecolor{backcolour}{rgb}{0.95,0.95,0.92}

% Configuración de la portada
\titleformat{\section}[block]{\normalfont\huge\bfseries}{\thesection}{1em}{}
\geometry{a4paper, margin=1in}

% Definir comandos para información repetitiva
\newcommand{\logoInstitucion}{recursos/logotipo_ipn.png} % Reemplaza con el nombre del archivo de tu logo
\newcommand{\logoUniversidad}{recursos/EscudoESCOM.png} % Reemplaza con el nombre del archivo de tu logo
\newcommand{\nombreInstituto}{Instituto Politecnico Nacional}
\newcommand{\facultad}{Escuela Superior de Computo}
\newcommand{\materia}{Tópicos Selectos de Algoritmos Bioinspirados}
\newcommand{\grupo}{7BM1}
\newcommand{\profesora}{Daniel Molina Pérez}
\newcommand{\periodo}{2025/02}

\newcommand{\alumnoA}{Carrillo Barreiro José Emiliano}
\newcommand{\alumnoB}{Martinez Ayala Gerardo}
\newcommand{\alumnoC}{Robles Otero José Ángel}
\newcommand{\alumnoD}{Vásquez Morales Haniel Ulises}
\newcommand{\alumnos}{Carrillo J., Martinex G., Robles J. \& Vásquez H.}

% Definir nuevas listas para información repetitiva
\newlist{bracketed}{enumerate}{1}
\setlist[bracketed,1]{label={[{\arabic*}]},left=0pt}

% Configuración del encabezado y pie de página
\pagestyle{fancy}
\fancyhf{} % Limpia todos los campos

% Encabezado
\fancyhead[C]{\textbf{\nombreInstituto\\ \facultad}}
\fancyhead[L]{\includegraphics[width=1cm]{recursos/logotipo_ipn.png}}
\fancyhead[R]{\includegraphics[width=1.5cm]{recursos/EscudoESCOM.png}}

% Pie de página
\fancyfoot[C]{\materia||\grupo\\ \alumnos\\ \thepage}
% Configuración del grosor de la línea en el encabezado y pie de página
\renewcommand{\headrulewidth}{0.4pt}
\renewcommand{\footrulewidth}{0.4pt}

% Ajuste del tamaño del encabezado
\setlength{\headheight}{34.0845pt}
\addtolength{\topmargin}{-22.0845pt}

\fancypagestyle{plain}{%
    \fancyhf{} % Borra encabezado y pie de página por defecto
    \fancyhead[C]{\textbf{\nombreInstituto\\ \facultad}}
    \fancyhead[L]{\includegraphics[width=1cm]{recursos/logotipo_ipn.png}}
    \fancyhead[R]{\includegraphics[width=1.5cm]{recursos/EscudoESCOM.png}}
    \fancyfoot[C]{\materia||\grupo\\ \alumnos\\ \thepage}
    \renewcommand{\headrulewidth}{0.4pt}
    \renewcommand{\footrulewidth}{0.4pt}
}


% Configuración de lstlisting para Python
\lstdefinestyle{mystyle}{
    backgroundcolor=\color{backcolour},
    commentstyle=\color{codegreen},
    keywordstyle=\color{codepurple},
    numberstyle=\tiny\color{codegray},
    stringstyle=\color{codered},
    basicstyle=\ttfamily\small,
    breakatwhitespace=false,
    breaklines=true,
    captionpos=b,
    keepspaces=true,
    numbers=left,
    numbersep=5pt,
    showspaces=false,
    showstringspaces=false,
    showtabs=false,
    tabsize=2
}

\lstdefinestyle{custompython}{
    language=Python,
    basicstyle=\small\ttfamily,
    keywordstyle=\color{blue},
    stringstyle=\color{green},
    commentstyle=\color{red},
    numbers=left,
    numberstyle=\tiny\color{gray},
    stepnumber=1,
    numbersep=5pt,
    backgroundcolor=\color{white},
    frame=single,
    captionpos=b,
    breaklines=true,
    breakatwhitespace=false,
    showstringspaces=false,
    morekeywords={import,plt,np,ws,y,gradient}
}

% Definición de estilo para C++
\lstdefinestyle{cppstyle}{
    language=C++,
    basicstyle=\small\ttfamily,
    keywordstyle=\color{blue},
    commentstyle=\color{green!40!black},
    stringstyle=\color{orange},
    showstringspaces=false,
    breaklines=true,
    breakatwhitespace=true,
    numbers=left,
    numberstyle=\tiny,
    stepnumber=1,
    numbersep=5pt,
    frame=single,
    captionpos=b,
    backgroundcolor=\color{gray!5},
    xleftmargin=0.5cm,
    xrightmargin=0.5cm
}

% Definición de estilo para MATLAB
\lstdefinestyle{matlabstyle}{
    language=Matlab,
    basicstyle=\small\ttfamily,
    keywordstyle=\color{blue},
    commentstyle=\color{green!40!black},
    stringstyle=\color{purple!50!blue!50!white},
    showstringspaces=false,
    breaklines=true,
    breakatwhitespace=true,
    numbers=left,
    numberstyle=\tiny,
    stepnumber=1,
    numbersep=5pt,
    frame=single,
    captionpos=b,
    backgroundcolor=\color{gray!5},
    xleftmargin=0.5cm,
    xrightmargin=0.5cm
}

%Definicion de estilo para Python
\lstdefinestyle{pythonstyle}{
    language=Python,
    basicstyle=\small\ttfamily,
    keywordstyle=\color{red!80!black},
    commentstyle=\color{gray!40!white},
    stringstyle=\color{yellow!30!orange!80!white},
    showstringspaces=false,
    breaklines=true,
    breakatwhitespace=true,
    numbers=left,
    numberstyle=\tiny,
    stepnumber=1,
    numbersep=5pt,
    frame=single,
    captionpos=b,
    backgroundcolor=\color{gray!5},
    xleftmargin=0.5cm,
    xrightmargin=0.5cm
}

% Personalización del índice de listados
\newcommand{\listcppname}{Índice de Listados de C++}
\newcommand{\listmatlabname}{Índice de Listados de MATLAB}
\newcommand{\listpythonname}{Índice de Listados de Python}
\newlistof{cpplistings}{lolcpp}{\listcppname}
\newlistof{matlablistings}{lolmatlab}{\listmatlabname}
\newlistof{pythonlistings}{lolpython}{\listpythonname}

% Configuración para los listados de C++
\lstset{style=cppstyle}
\newcommand{\cpplisting}[1]{%
    \refstepcounter{cpplistings}%
    \addcontentsline{lolcpp}{cpplistings}{\protect\numberline{\thecpplistings}#1}\par}

% Configuración para los listados de MATLAB
\lstset{style=matlabstyle}
\newcommand{\matlablisting}[1]{%
    \refstepcounter{matlablistings}%
    \addcontentsline{lolmatlab}{matlablistings}{\protect\numberline{\thematlablistings}#1}\par}

% Configuración para los listados de Python
\lstset{style=pythonstyle}
\newcommand{\pythonlisting}[1]{%
    \refstepcounter{pythonlistings}%
    \addcontentsline{lolpython}{pythonlistings}{\protect\numberline{\thepythonlistings}#1}\par}

\begin{document}

% Portada
% portada
\begin{titlepage}
    \begin{center}
        \vspace*{1cm}

        \begin{tabular}{c@{\hspace{2cm}}c}
            \includegraphics[width=0.25\textwidth]{\logoInstitucion} &
            \includegraphics[width=0.4\textwidth]{\logoUniversidad}
        \end{tabular}

        \vspace{1cm}

        \textbf{\LARGE \nombreInstituto} \\
        \textbf{\Large \facultad} \\
        \vspace{0.5cm}
        \textbf{\large Materia: \materia} \\
        \textbf{\large Grupo: \grupo} \\
        \vspace{0.5cm}
        \textbf{\large Profesor: \profesora} \\
        \textbf{\large Periodo: \periodo} \\

        \vspace{0.75cm}

        \textbf{\LARGE Practica 01} \\
        \vspace{0.5cm}
        \textbf{\Large \textit{Maximizar Contraste en Imagenes Medicas.}} \\

        \vspace{0.3cm}

        \textbf{\large Realizado por:} \\
        \textbf{\large \alumnoA \\ \alumnoB \\ \alumnoC \\ \alumnoD}

        % Bloque de Resumen y Abstract
        \begin{minipage}{0.8\textwidth}
            \textbf{Abstract:}\\[0.3cm]
            This report describes the design, implementation, and evaluation of a Genetic Algorithm (GA) aimed at minimizing multimodal functions. Two benchmark functions—Langermann and Drop-Wave—serve as test cases. The GA employs tournament selection, Simulated Binary Crossover (SBX) with boundary handling, and polynomial mutation with boundaries. In addition to detailing the algorithm’s components, the report outlines the experimental setup, visualization techniques, and potential avenues for future improvements.
        \end{minipage}

        \begin{minipage}{0.8\textwidth}
            \textbf{Resumen:}\\[0.3cm]
            En este reporte se describe el diseño, implementacion y evaluacion de un Algoritmo Genetico (GA) adecuado a minimizar funciones multi-modales. Con dos funciones de evaluación comparativa (Benchmark functions)--\textit{Langermann} y \textit{Drop-Wave}-- sirven como ejemplificacion de casos de uso. El GA implementa: \textit{Selección por Torneo}, \textit{Simulated Binary Crossover (SBX)} con manejo de limites, \textit{Mutación Polinomial} con uso de cotas y \textit{Sustitución Extinitiva con Elitismo}. Ademas de detallar los componentes de los algoritmos, el reporte añade los valores de los parametros utilizados, tecnicas de visualizacion y potenciales caminos para su mejora continua.
        \end{minipage}
        
        \vspace{0.3cm}

        \textbf{\large Fecha: \today}

    \end{center}
\end{titlepage}


% Índice
\input{secciones/indices.tex}

%Capitulos
\chapter{Introducción}
Los algoritmos genéticos (AG) son una clase de algoritmos bioinspirados ampliamente utilizados para resolver problemas de optimización y búsqueda. Inspirados en el proceso de evolución natural, estos algoritmos imitan mecanismos biológicos como la selección, el cruzamiento y la mutación para explorar el espacio de soluciones y encontrar resultados óptimos. En este proyecto se implementa un AG para la minimización de funciones benchmark, en particular las funciones Langermann y Drop-Wave, con el objetivo de demostrar y analizar la efectividad de técnicas como la selección por torneo, el cruzamiento Simulated Binary Crossover (SBX) y la mutación polinomial.

\chapter{Objetivos del Proyecto}

\section{Objetivo general}
Desarrollar e implementar un algoritmo genético para la resolución de un problema de optimización restringida en el ámbito financiero, con el fin de determinar la cartera de inversión óptima bajo diferentes criterios de maximización de retorno y control de riesgo.

\section{Objetivo del inciso uno}
Determinar la asignación óptima de capital en una cartera compuesta por seis acciones, maximizando el retorno esperado bajo la restricción de no invertir más del 40% en una sola acción, sin considerar el riesgo asociado.

\section{Objetivo inciso dos}
Diseñar una estrategia de inversión que minimice el riesgo asociado a la cartera, garantizando un rendimiento esperado mínimo del 35% y respetando la condición de no invertir más del 40% en una sola acción.

\section{Objetivo inciso tres}
Identificar la combinación de inversión que maximice el rendimiento esperado de la cartera sin sobrepasar un nivel de riesgo del 20%, manteniendo la restricción de concentración del 40% por acción.

\chapter{Metodolog\'ia}

\section{Inicializaci\'on de la Poblaci\'on}
La poblaci\'on inicial se genera de forma uniforme a lo largo del espacio de b\'usqueda, definido por l\'imites inferiores y superiores para cada variable.

\subsection*{Objetivo}
Garantizar que la b\'usqueda comience explorando de manera equitativa todas las regiones posibles, evitando sesgos que puedan limitar la diversidad de soluciones iniciales.

\subsection*{Implementaci\'on}
Se utiliza la funci\'on \texttt{initialize\_population}, la cual emplea m\'etodos de generaci\'on aleatoria (por ejemplo, la funci\'on \texttt{np.random.uniform} de NumPy) para crear un conjunto de individuos.

\subsection*{Ventajas}
\begin{itemize}
    \item Permite cubrir todo el rango definido para cada variable.
    \item Aumenta la probabilidad de encontrar regiones prometedoras del espacio de soluciones desde el inicio.
\end{itemize}

\section{Evaluaci\'on de Fitness}

Cada individuo generado se eval\'ua mediante la funci\'on objetivo, la cual determina qu\'e tan buena es la soluci\'on propuesta.

\subsection*{Funciones Utilizadas}
\begin{itemize}
    \item \textbf{Langermann}: Es una funci\'on multimodal que combina componentes cosenoidales y exponenciales, generando m\'ultiples \'optimos locales.
    \item \textbf{Drop-Wave}: Una funci\'on bidimensional con una superficie ondulada, usada para analizar el comportamiento del algoritmo en entornos con m\'ultiples picos y valles.
\end{itemize}

\subsection*{Proceso}
Se calcula el valor de fitness para cada individuo (por ejemplo, evaluando $f(x_1, x_2)$) y se almacena dicho valor para posteriores comparaciones.

\subsection*{Importancia}
La evaluaci\'on correcta del fitness es crucial, ya que determina la selecci\'on de individuos y, por ende, el rumbo de la evoluci\'on poblacional.

\section{Selecci\'on por Torneo}

Para elegir los padres que generar\'an la siguiente generaci\'on se utiliza un m\'etodo de selecci\'on por torneo.

\subsection*{Mecanismo}
\begin{itemize}
    \item Se forman m\'ultiples grupos (torneos) de individuos seleccionados al azar.
    \item En cada grupo se compara el fitness de los participantes y se selecciona al individuo con el mejor desempe\~no.
\end{itemize}

\subsection*{Implementaci\'on Vectorizada}
La funci\'on \texttt{vectorized\_tournament\_selection} realiza este proceso de forma eficiente, aprovechando operaciones vectorizadas de NumPy.

\subsection*{Beneficios}
\begin{itemize}
    \item Favorece la selecci\'on de soluciones de alta calidad sin descartar por completo la diversidad poblacional.
    \item Permite controlar la presi\'on selectiva mediante el tama\~no del torneo.
\end{itemize}

\section{Cruzamiento con SBX}

El operador de cruzamiento se implementa mediante el m\'etodo SBX (Simulated Binary Crossover).

\subsection*{Proceso del SBX}
\begin{itemize}
    \item A partir de dos padres, se genera un n\'umero aleatorio $u$ y se calcula un par\'ametro $\beta$ que determina la dispersi\'on de los descendientes respecto a los padres.
    \item Se generan dos hijos combinando linealmente los valores de los padres.
\end{itemize}

\subsection*{Ajuste de L\'imites}
Se incorpora un mecanismo en \texttt{sbx\_crossover\_with\_boundaries} que garantiza que los hijos resultantes se mantengan dentro de los l\'imites predefinidos.

\subsection*{Ventajas}
\begin{itemize}
    \item Promueve la creaci\'on de soluciones intermedias que pueden explotar la informaci\'on gen\'etica de ambos padres.
    \item Ayuda a preservar la diversidad en la poblaci\'on.
\end{itemize}

\section{Mutaci\'on Polinomial}

Para introducir variabilidad y explorar nuevas regiones del espacio de b\'usqueda, se aplica la mutaci\'on polinomial.

\subsection*{Mecanismo de la Mutaci\'on}
\begin{itemize}
    \item Cada gen de un individuo tiene una probabilidad definida de sufrir una mutaci\'on.
    \item Se usa una distribuci\'on polinomial, controlada por el par\'ametro $\eta_{\text{mut}}$.
\end{itemize}

\subsection*{Consideraciones de L\'imites}
La mutaci\'on se aplica respetando los l\'imites definidos para cada variable mediante la funci\'on \texttt{polynomial\_mutation\_with\_boundaries}.

\subsection*{Beneficios}
\begin{itemize}
    \item Introduce peque\~nas variaciones que pueden conducir a la exploraci\'on de nuevas soluciones.
    \item Previene la convergencia prematura al mantener la diversidad gen\'etica.
\end{itemize}

\section{Elitismo y Ciclo Evolutivo}

El proceso evolutivo se estructura en ciclos o generaciones.

\subsection*{Elitismo}
\begin{itemize}
    \item Se retiene el mejor individuo de la generaci\'on actual y se garantiza su inclusi\'on en la siguiente generaci\'on.
    \item Esto asegura que la calidad de la soluci\'on nunca empeore a lo largo de las generaciones.
\end{itemize}

\subsection*{Ciclo Evolutivo}
\begin{itemize}
    \item Cada generaci\'on incluye la selecci\'on, el cruzamiento, la mutaci\'on y la incorporaci\'on del individuo de \'elite.
    \item La evoluci\'on se repite durante un n\'umero predefinido de generaciones.
\end{itemize}

\subsection*{Registro y An\'alisis}
\begin{itemize}
    \item Se almacena el historial del fitness y de las mejores soluciones.
    \item Esto facilita el an\'alisis del comportamiento del algoritmo y la generaci\'on de visualizaciones.
\end{itemize}

\chapter{Resultados y Discusi\'on}

Durante la ejecuci\'on del algoritmo se realizaron m\'ultiples corridas completas para cada funci\'on objetivo (Langermann y Drop-Wave), lo que permiti\'o evaluar la estabilidad y eficiencia del m\'etodo. Los resultados se agruparon en res\'umenes globales\footnote{Se anexan tablas de los resumenes independientes de cada ejecución en la seccion de resumenes e historiales correspondiente a cada función dentro del capitulo de anexos.}, donde se registraron indicadores clave, los cuales sencuentran dentro de la siguientes tablas:
\begin{longtable}{lrrrrrrr}
\caption{Problema 03: resumen\_global\_corridas}\label{tab:resumen_global_corridas} \\
\toprule
Indicador & x0 & x1 & x2 & x3 & x4 & x5 & Fitness \\
\midrule
\endfirsthead
\toprule
Indicador & x0 & x1 & x2 & x3 & x4 & x5 & Fitness \\
\midrule
\endhead
\midrule
\multicolumn{8}{r}{Continued on next page} \\
\midrule
\endfoot
\bottomrule
\endlastfoot
Mejor (Fitness) & 0.2971 & 0.1112 & 0.0351 & 0.2159 & 0.1974 & 0.1433 & 0.0028 \\
Peor (Fitness) & 0.1291 & 0.2206 & 0.0217 & 0.3507 & 0.0449 & 0.2331 & 0.0045 \\
Media & 0.1934 & 0.1866 & 0.0286 & 0.264 & 0.1439 & 0.1835 & 0.0037 \\
Desv. Estándar & 0.0563 & 0.0499 & 0.0174 & 0.0551 & 0.0945 & 0.0484 & 0.0006 \\
\end{longtable}
\begin{longtable}{lrrrrrrr}
\caption{Problema 03: resumen\_global\_corridas}\label{tab:resumen_global_corridas} \\
\toprule
Indicador & x0 & x1 & x2 & x3 & x4 & x5 & Fitness \\
\midrule
\endfirsthead
\toprule
Indicador & x0 & x1 & x2 & x3 & x4 & x5 & Fitness \\
\midrule
\endhead
\midrule
\multicolumn{8}{r}{Continued on next page} \\
\midrule
\endfoot
\bottomrule
\endlastfoot
Mejor (Fitness) & 0.2971 & 0.1112 & 0.0351 & 0.2159 & 0.1974 & 0.1433 & 0.0028 \\
Peor (Fitness) & 0.1291 & 0.2206 & 0.0217 & 0.3507 & 0.0449 & 0.2331 & 0.0045 \\
Media & 0.1934 & 0.1866 & 0.0286 & 0.264 & 0.1439 & 0.1835 & 0.0037 \\
Desv. Estándar & 0.0563 & 0.0499 & 0.0174 & 0.0551 & 0.0945 & 0.0484 & 0.0006 \\
\end{longtable}

Donde cada Indicador Representa lo siguiente:
\begin{itemize}
    \item \textbf{Mejor (Fitness):} Representa la soluci\'on con el valor de fitness m\'inimo obtenido en todas las corridas.
    \item \textbf{Peor (Fitness):} Indica la soluci\'on con el mayor valor de fitness, sirviendo como referencia de la variabilidad en la b\'usqueda.
    \item \textbf{Media:} Es el promedio de los valores de fitness de la mejor soluci\'on de cada corrida, ofreciendo una visi\'on global del desempe\~no del algoritmo.
    \item \textbf{Desv. Est\'andar:} Mide la dispersi\'on de los valores de fitness entre las corridas, reflejando la estabilidad y consistencia del proceso evolutivo.
\end{itemize}

\section{An\'alisis de los Resultados}

\subsection{Consistencia y Robustez}
Los res\'umenes globales muestran que, a lo largo de las corridas, el algoritmo tiende a converger de manera consistente hacia soluciones de alta calidad. Una baja desviaci\'on est\'andar en los valores de fitness sugiere que el proceso evolutivo es robusto y no depende en exceso de la aleatoriedad inherente a los operadores gen\'eticos. Esto es crucial para problemas de optimizaci\'on, ya que garantiza que la metodolog\'ia aplicada es reproducible y confiable.

\subsection{Comparaci\'on Entre Funciones}
\subsubsection{Funci\'on Langermann}
El resumen global para Langermann indica que el algoritmo fue capaz de identificar una soluci\'on cercana al \'optimo global, a pesar de la presencia de m\'ultiples \'optimos locales debido a la naturaleza multimodal de la funci\'on. El valor de ``Mejor (Fitness)'' obtenido se sit\'ua en un rango competitivo, y la media de las corridas respalda la eficacia del operador SBX y la mutaci\'on polinomial para explorar el espacio de soluciones.

\subsubsection{Funci\'on Drop-Wave}
Para la funci\'on Drop-Wave, que presenta una superficie de b\'usqueda ondulada, los resultados globales evidencian una convergencia hacia regiones con valores de fitness bajos, a pesar de la complejidad inherente a la topolog\'ia de la funci\'on. Los indicadores de ``Peor (Fitness)'' y ``Desv. Est\'andar'' muestran que, aunque existen ciertas variaciones entre corridas, el mecanismo de elitismo y la correcta aplicaci\'on de los operadores gen\'eticos ayudan a mitigar posibles desviaciones, logrando resultados consistentes.

\subsection{Evoluci\'on del Fitness y Visualizaciones}
Las gr\'aficas de la evoluci\'on del fitness, tanto en su forma original como normalizada, permiten observar el progreso generacional. Se aprecia una clara tendencia a la mejora, donde la mayor\'ia de las corridas muestran una reducci\'on significativa del valor de fitness a medida que avanzan las generaciones.

A continuación se dejan las gráficas de la evolución del fitness:
\begin{figure}[H]
    \centering
    \includegraphics[width=\textwidth]{secciones/tablas/langermann/evolucion_fitness_langermann.png}
    \caption{Evolución del \text{fitness} de la función de Langermann}
    \label{fig:fitness_langermann}
\end{figure}

\begin{figure}[H]
    \centering
    \includegraphics[width=\textwidth]{secciones/tablas/drop_wave/evolucion_fitness_drop_wave.png}
    \caption{Evolución del \text{fitness} de la función Drop Wave}
    \label{fig:fitness_drop_wave}
\end{figure}

Adem\'as, la visualizaci\'on 3D de la superficie de la funci\'on (para casos bidimensionales) complementa el an\'alisis, ya que permite ver la distribuci\'on espacial de las mejores soluciones encontradas en cada corrida. Esto confirma visualmente la capacidad del algoritmo para explorar eficazmente el espacio de b\'usqueda y concentrarse en las regiones prometedoras.

A continuación se dejan la visualización 3D de la superficie de las funciones y su proyección en el plano de las variables optimizadas:
\begin{figure}[H]
    \centering
    \includegraphics[width=\textwidth]{secciones/tablas/langermann/surface_3d_langermann.png}
    \caption{Representación y superficie de la función de Langermann}
    \label{fig:surface_langermann}
\end{figure}

\begin{figure}[H]
    \centering
    \includegraphics[width=\textwidth]{secciones/tablas/drop_wave/surface_3d_drop_wave.png}
    \caption{Representación y superficie de la función Drop Wave}
    \label{fig:surface_drop_wave}
\end{figure}

\section{Discusión Resultados}

\begin{itemize}
    \item \textbf{Eficacia del Algoritmo:} Los indicadores globales extra\'idos de los CSV demuestran que el algoritmo gen\'etico es capaz de acercarse a la soluci\'on \'optima, manteniendo una evoluci\'on progresiva y consistente en la reducci\'on del valor de fitness.
    
    \item \textbf{Diversidad y Convergencia:} La aplicaci\'on de operadores de selecci\'on, cruzamiento y mutaci\'on, junto con el mecanismo de elitismo, garantiza un equilibrio entre la exploraci\'on y la explotaci\'on del espacio de b\'usqueda. Esto se refleja en la baja variabilidad entre corridas, lo que es un indicativo de la estabilidad del proceso.
    
    \item \textbf{Potencial de Adaptaci\'on:} La estructura modular y la robustez mostrada por los resultados permiten considerar la posibilidad de aplicar este marco a otros problemas de optimizaci\'on, incluso aquellos con mayores dimensiones o con funciones objetivo de mayor complejidad.
\end{itemize}


\chapter{Implementaci\'on}

El proyecto se ha desarrollado siguiendo una arquitectura modular\footnote{Se puede encontrar el codigo perteneciente a cada funcion dentro de} que permite separar claramente las distintas funcionalidades y facilita tanto el mantenimiento como la ampliaci\'on futura. A continuaci\'on, se detallan los principales componentes y c\'omo se integran en el sistema:

\section{Funciones Objetivo}

\subsection{Descripci\'on}
Las funciones objetivo definen el problema a optimizar. En este proyecto se incluyen dos funciones:
\begin{itemize}
    \item \textbf{Tasa de retorno del portafolio}: Una funci\'on que nos permite calcular la ganancia esperada de un portafolio de inversion considerando las ganancias individuales de cada instrumento.
    \item \textbf{Varianza del portafolio}: Una funci\'on que nos permite calcular el riesgo de un portafolio de inversion considerando la covarianza de nuestros instrumentos de inversion.
\end{itemize}

\subsection{Implementaci\'on}
Se abordaron tres problemas de inversión, cada uno con sus respectivas limitaciones. En los tres casos se establecieron límites para todos los instrumentos de inversión en el intervalo \([0, 10]\). 

En el primer problema, se buscó maximizar el beneficio del portafolio, es decir, maximizar la tasa de retorno. En el segundo problema, el objetivo fue minimizar la varianza del portafolio, manteniendo una ganancia mínima del \(35\%\); por lo tanto, se incorporó esta restricción al modelo. Finalmente, en el tercer problema se volvió a buscar la maximización de la tasa de retorno del portafolio, pero añadiendo la restricción de considerar un riesgo menor al \(20\%\).

\section{M\'odulos del Algoritmo Gen\'etico}

El n\'ucleo del algoritmo gen\'etico se distribuye en varios m\'odulos:

\subsection{Inicializaci\'on}
\textbf{Funci\'on}: \texttt{initialize\_population}  
\textbf{Ubicaci\'on}: \texttt{libs/auxiliaries\_functions.py}  
\textbf{Descripci\'on}: Genera la poblaci\'on inicial de manera uniforme en el espacio de b\'usqueda.

\subsection{Selecci\'on}
\textbf{Funci\'on}: \texttt{vectorized\_tournament\_selection}  
\textbf{Ubicaci\'on}: \texttt{libs/selection.py}  
\textbf{Descripci\'on}: Se usa un enfoque de torneos para la selecci\'on de padres, empleando operaciones vectorizadas con NumPy.

\subsection{Cruzamiento}
\textbf{Funciones}: \texttt{sbx\_crossover}, \texttt{sbx\_crossover\_with\_boundaries}  
\textbf{Ubicaci\'on}: \texttt{libs/crossover.py}  
\textbf{Descripci\'on}: Implementa el operador SBX (Simulated Binary Crossover) con y sin control de l\'imites.

\subsection{Mutaci\'on}
\textbf{Funciones}: \texttt{polynomial\_mutation}, \texttt{polynomial\_mutation\_with\_boundaries}  
\textbf{Ubicaci\'on}: \texttt{libs/mutation.py}  
\textbf{Descripci\'on}: Aplica mutaci\'on polinomial, con control opcional de l\'imites para mantener la viabilidad de las soluciones.

\subsection{Aplicacion de Restricciones}
\textbf{Función}: \texttt{evaluate\_individuals\_with\_constraints} 
\textbf{Ubicaci\'on}: \texttt{libs/auxiliaries\_functions.py}  
\textbf{Descripción}: Verifica el cumplimiento de las restricciones impuestas por el problema. Cuando se violan restricciones, se aplica un mecanismo de penalización para ajustar el valor del fitness, de modo que las soluciones inviables tengan menor probabilidad de ser seleccionadas.

\subsection{Ejecuci\'on del Algoritmo}
\textbf{Funci\'on}: \texttt{genetic\_algorithm}  
\textbf{Ubicaci\'on}: \texttt{AG.py}  
\textbf{Descripci\'on}: Gestiona el ciclo evolutivo completo del algoritmo gen\'etico.

\section{Visualizaci\'on y Almacenamiento}

\subsection{Visualizaci\'on}
\textbf{M\'odulo}: \texttt{libs/plot.py}  
\textbf{Funciones}: \texttt{plot\_evolucion\_fitness}, \texttt{plot\_surface\_3d}  
\textbf{Descripci\'on}: Permite analizar la evoluci\'on del fitness y visualizar la superficie de las funciones objetivo.

\subsection{Almacenamiento}
\textbf{Estructura de Carpetas}:
\begin{itemize}
    \item Directorio \texttt{outputs} organizado en subcarpetas por problema.
    \item Historiales en archivos CSV con datos de fitness y variables.
    \item Res\'umenes estad\'isticos de cada corrida.
\end{itemize}

\textbf{Integraci\'on}: \texttt{main\_script.py} ejecuta el algoritmo para cada funci\'on definida en \texttt{AG\_confs.py}.

\textbf{Escalabilidad}: La arquitectura modular permite agregar nuevas funciones objetivo y modificar operadores gen\'eticos sin afectar la estructura base.
\chapter{Conclusiones}
El proyecto demuestra la efectividad de los algoritmos genéticos en la optimización de funciones complejas. La implementación de técnicas avanzadas, como el cruzamiento SBX y la mutación polinomial, combinada con una estrategia de selección por torneo, ha permitido explorar de manera eficiente el espacio de soluciones y mejorar progresivamente el fitness de la población. Las herramientas de visualización y el almacenamiento de resultados facilitan el análisis del comportamiento del algoritmo y ofrecen una base sólida para futuras mejoras o aplicaciones a problemas más complejos.

Además, el algoritmo genético desarrollado constituye un marco sólido y escalable para la optimización de funciones complejas. Los resultados obtenidos respaldan la viabilidad del enfoque y abren la puerta a futuras investigaciones, ya sea para afinar los parámetros del algoritmo o para extender su aplicación a problemas con mayores dimensiones o características más complejas. Esta base permite, además, la incorporación de mejoras y la adaptación del método a diferentes contextos, consolidando su utilidad en el ámbito de la optimización computacional.
\appendix
\chapter{GitHub}
Se anexa un enlace al repositorio donde se encuentran los codigos y una README que examnde la información aqui mostrada: \href{https://github.com/CARBAJE/Practica02-TSAB}{\textit{Click aqui para abrir el enlace al repositorio}}
funcion\chapter{Scripts}

\section{Archivo \texttt{main.py}}
\begin{lstlisting}[
  caption={Implementación de \texttt{main.py}},
  style=pythonstyle,
  basicstyle=\ttfamily\footnotesize
]
import os
import pandas as pd
from AG_confs import *

from AG import genetic_algorithm
from libs.plot import *

def main():
    # Crear carpetas de salida generales
    os.makedirs("outputs", exist_ok=True)
    
    for func_key, func_data in FUNCTIONS.items():
        f_obj = func_data["func"]
        lb = func_data["lb"]
        ub = func_data["ub"]
        func_name = func_data["name"]
        num_runs = func_data["num_runs"]
        
        # Carpetas especificas de cada funcion
        func_folder = f"outputs/{func_key}"
        os.makedirs(func_folder, exist_ok=True)
        hist_folder = os.path.join(func_folder, "historiales")
        res_folder = os.path.join(func_folder, "resumenes")
        os.makedirs(hist_folder, exist_ok=True)
        os.makedirs(res_folder, exist_ok=True)
        
        print(f"\n==============================================")
        print(f"  FUNCIoN: {func_name}")
        print(f"==============================================")
        
        all_runs_history = []
        best_solutions_all_runs = []  # Guardaremos los mejores individuos (x1, x2) de cada corrida
        best_values_across_runs = []  # Guardaremos el best_val (fitness) de cada corrida
        
        for run in range(num_runs):
            print(f"\nEjecucion {run+1}/{num_runs}")
            
            (best_sol, best_val,
             worst_sol, worst_val,
             avg_sol,  avg_val,
             std_val,
             best_fitness_history,
             best_x1_history,
             best_x2_history,
             population_final,
             fitness_final,
             best_solutions_over_time) = genetic_algorithm(
                 f_obj, lb, ub,
                 pop_size=POP_SIZE,
                 num_generations=NUM_GENERATIONS,
                 tournament_size=TOURNAMENT_SIZE,
                 crossover_prob=CROSSOVER_PROB,
                 eta_c=ETA_C,
                 mutation_prob=MUTATION_PROB,
                 eta_mut=ETA_MUT
             )
            
            # 1) Guardar historial
            df_historial = pd.DataFrame({
                "Generacion": np.arange(1, NUM_GENERATIONS + 1),
                "Mejor x1": best_x1_history,
                "Mejor x2": best_x2_history,
                "Mejor Fitness": best_fitness_history
            })
            historial_filename = os.path.join(hist_folder, f"historial_run_{run+1}.csv")
            df_historial.to_csv(historial_filename, index=False)
            
            # 2) Guardar resumen de la corrida
            data_resumen = [
                ["Mejor", best_sol[0], best_sol[1], best_val],
                ["Media", avg_sol[0], avg_sol[1], avg_val],
                ["Peor", worst_sol[0], worst_sol[1], worst_val],
                ["Desv. estandar", np.nan, np.nan, std_val]
            ]
            df_resumen = pd.DataFrame(data_resumen, columns=["Indicador", "x1", "x2", "Fitness"])
            resumen_filename = os.path.join(res_folder, f"resumen_run_{run+1}.csv")
            df_resumen.to_csv(resumen_filename, index=False)
            
            print(df_resumen.to_string(index=False))
            
            all_runs_history.append(best_fitness_history)
            best_solutions_all_runs.append(best_sol)
            best_values_across_runs.append(best_val)
        
        # ===========================================
        #       RESUMEN GLOBAL DE LAS CORRIDAS
        # ===========================================
        best_values_arr = np.array(best_values_across_runs)
        solutions_arr = np.array(best_solutions_all_runs)  # Cada fila: [x1, x2] del mejor individuo de cada corrida
        
        # Para el "Mejor" y "Peor", buscamos el indice de la corrida con minimo y maximo fitness
        min_index = np.argmin(best_values_arr)
        max_index = np.argmax(best_values_arr)
        
        data_global = [
            ["Mejor (Fitness)", solutions_arr[min_index, 0], solutions_arr[min_index, 1], best_values_arr[min_index]],
            ["Peor (Fitness)", solutions_arr[max_index, 0], solutions_arr[max_index, 1], best_values_arr[max_index]],
            ["Media", np.mean(solutions_arr[:, 0]), np.mean(solutions_arr[:, 1]), np.mean(best_values_arr)],
            ["Desv. Estandar", np.std(solutions_arr[:, 0]), np.std(solutions_arr[:, 1]), np.std(best_values_arr)]
        ]
        df_global = pd.DataFrame(data_global, columns=["Indicador", "x1", "x2", "Fitness"])
        
        global_filename = os.path.join(res_folder, "resumen_global_corridas.csv")
        df_global.to_csv(global_filename, index=False)
        
        # (Opcional) Graficar evolucion del fitness de todas las corridas
        plot_evolucion_fitness(all_runs_history, func_key, func_name)
        
        # (Opcional) Graficar superficie 3D si la funcion es de 2 variables
        if len(lb) == 2:
            plot_surface_3d(f_obj, lb, ub, best_solutions_all_runs, func_key, func_name)

if __name__ == "__main__":
    main()

\end{lstlisting}

\section{Archivo \texttt{AG\_confs.py}}
\begin{lstlisting}[
  caption={Implementación de \texttt{AG\_confs.py}},
  style=pythonstyle,
  basicstyle=\ttfamily\footnotesize
]
import numpy as np
from libs.functions import langermann, drop_wave
# ---------------------------
# Parametros del algoritmo
# ---------------------------
POP_SIZE = 100            # Numero de individuos en la poblacion
NUM_GENERATIONS = 200     # Numero de generaciones
NUM_RUNS = 10             # Numero de ejecuciones completas (ciclos)

# Parametros de la funcion de Langermann
a = np.array([3, 5, 2, 1, 7])
b = np.array([5, 2, 1, 4, 9])
c = np.array([1, 2, 5, 2, 3])

# Parametros del torneo
TOURNAMENT_SIZE = 3  # Numero de individuos participantes en cada torneo

# Parametros del cruzamiento SBX
CROSSOVER_PROB = 0.9  # Probabilidad de aplicar cruzamiento
ETA_C = 15            # indice de distribucion para SBX

# Parametros de la mutacion polinomial
MUTATION_PROB = 10.0 / 2  # Probabilidad de mutar cada gen
ETA_MUT = 20              # indice de distribucion para mutacion polinomial

best_solutions_list = [] 
all_runs_history = []  # Para graficar luego

FUNCTIONS = {
    "langermann": {
        "func": langermann,
        "lb": np.array([0, 0]),
        "ub": np.array([10, 10]),
        "name": "Langermann",
        "num_runs": NUM_RUNS
    },
    "drop_wave": {
        "func": drop_wave,
        "lb": np.array([-5.12, -5.12]),
        "ub": np.array([5.12, 5.12]),
        "name": "Drop-Wave",
        "num_runs": NUM_RUNS
    }
}
\end{lstlisting}

\section{Archivo \texttt{AG.py}}
\begin{lstlisting}[
  caption={Implementación de \texttt{AG.py}},
  style=pythonstyle,
  basicstyle=\ttfamily\footnotesize
]
from AG_confs import *
from libs.selection import vectorized_tournament_selection
from libs.crossover import sbx_crossover_with_boundaries
from libs.mutation import polynomial_mutation_with_boundaries
from libs.auxiliaries_functions import initialize_population


# ---------------------------
# Funcion principal del GA
# ---------------------------
def genetic_algorithm(objective_func, lower_bound, upper_bound,
                      pop_size=POP_SIZE, num_generations=NUM_GENERATIONS,
                      tournament_size=TOURNAMENT_SIZE,
                      crossover_prob=CROSSOVER_PROB, eta_c=ETA_C,
                      mutation_prob=MUTATION_PROB, eta_mut=ETA_MUT):
    """
    Ejecuta el GA para la funcion objetivo dada y retorna:
      - best_solution, best_value
      - worst_solution, worst_value
      - avg_solution, avg_value
      - std_value (fitness)
      - best_fitness_history, best_x1_history, best_x2_history
      - population (final), fitness (final)
      - best_solutions_over_time (para animaciones)
    """
    num_variables = len(lower_bound)
    
    # 1) Inicializar poblacion
    population = initialize_population(pop_size, num_variables, lower_bound, upper_bound)
    fitness = np.array([objective_func(ind) for ind in population])
    
    best_fitness_history = []
    best_x1_history = []
    best_x2_history = []
    
    # Para animacion: almacenamos el mejor (x1, x2) en cada generacion
    best_solutions_over_time = np.zeros((num_generations, num_variables))
    
    for gen in range(num_generations):
        # Elitismo: guardar el mejor de la generacion actual
        best_index = np.argmin(fitness)
        best_fitness = fitness[best_index]
        elite = population[best_index].copy()
        
        best_fitness_history.append(best_fitness)
        best_x1_history.append(elite[0])
        best_x2_history.append(elite[1])
        best_solutions_over_time[gen, :] = elite
        
        new_population = []
        
        # Numero de padres necesarios (2 por cada par a generar)
        num_parents_needed = 2 * (pop_size - 1)
        winners, _ = vectorized_tournament_selection(fitness, num_parents_needed,
                                                     tournament_size, len(population),
                                                     unique_in_column=True, unique_in_row=False)
        
        # Generar un valor global para el crossover y otro para la mutacion (para toda la generacion)
        global_u = np.random.rand()
        global_r = np.random.rand()
        
        # Generar nueva poblacion
        for i in range(0, len(winners), 2):
            parent1 = population[winners[i]].copy()
            if i + 1 < len(winners):
                parent2 = population[winners[i+1]].copy()
            else:
                parent2 = parent1.copy()
            
            # Cruzamiento SBX usando el mismo u para todas las variables del cruce
            child1, child2 = sbx_crossover_with_boundaries(
                parent1, parent2, lower_bound, upper_bound,
                eta_c, crossover_prob, use_global_u=True, global_u=global_u
            )
            # Mutacion polinomial usando el mismo r para todas las variables del individuo
            child1 = polynomial_mutation_with_boundaries(
                child1, lower_bound, upper_bound,
                mutation_prob, eta_mut, use_global_r=True, global_r=global_r
            )
            child2 = polynomial_mutation_with_boundaries(
                child2, lower_bound, upper_bound,
                mutation_prob, eta_mut, use_global_r=True, global_r=global_r
            )
            
            new_population.append(child1)
            if len(new_population) < pop_size - 1:
                new_population.append(child2)
        
        # Convertir a array y evaluar el fitness de la nueva poblacion
        new_population = np.array(new_population)
        new_fitness = np.array([objective_func(ind) for ind in new_population])
        
        # Incorporar el individuo elite (elitismo)
        new_population = np.vstack([new_population, elite])
        new_fitness = np.append(new_fitness, best_fitness)
        
        # Actualizar la poblacion y su fitness para la siguiente generacion
        population = new_population.copy()
        fitness = new_fitness.copy()
    
    # Calcular estadisticas finales
    best_index = np.argmin(fitness)
    worst_index = np.argmax(fitness)
    best_solution = population[best_index]
    best_value = fitness[best_index]
    worst_solution = population[worst_index]
    worst_value = fitness[worst_index]
    avg_solution = np.mean(population, axis=0)
    avg_value = np.mean(fitness)
    std_value = np.std(fitness)
    
    return (best_solution, best_value,
            worst_solution, worst_value,
            avg_solution, avg_value,
            std_value,
            best_fitness_history,
            best_x1_history,
            best_x2_history,
            population,
            fitness,
            best_solutions_over_time)
\end{lstlisting}

\section{Archivo \texttt{selection.py}}
\begin{lstlisting}[
  caption={Implementación de \texttt{selection.py}},
  style=pythonstyle,
  basicstyle=\ttfamily\footnotesize
]
import numpy as np

def vectorized_tournament_selection(fitness, num_tournaments, tournament_size, pop_size,
                                    unique_in_column=True, unique_in_row=False):
    """
    Genera una matriz de torneos de forma vectorizada y retorna, para cada torneo,
    el indice del individuo ganador (el de menor fitness).
    
    Args:
      - fitness: array con los fitness de la poblacion (longitud = pop_size).
      - num_tournaments: numero de torneos a realizar (por ejemplo, el numero total
                         de selecciones de padres requeridas en la generacion).
      - tournament_size: numero de individuos que participan en cada torneo.
      - pop_size: tamano de la poblacion.
      - unique_in_column: si True, para cada posicion (columna) se eligen candidatos sin
                          repeticion entre torneos.
      - unique_in_row: si True, en cada torneo (fila) los candidatos seran unicos.
                    (Por defecto se permite repetir en la fila).
    
    Returns:
      - winners: array de indices ganadores (uno por torneo).
      - tournament_matrix: la matriz de candidatos (de tamano [num_tournaments x tournament_size]).
    """
    if unique_in_row:
        # Para cada torneo (fila), muestreamos sin reemplazo (cada fila es unica)
        tournament_matrix = np.array([np.random.choice(pop_size, size=tournament_size, replace=False)
                                      for _ in range(num_tournaments)])
    else:
        # Permitir repeticion en la fila, pero controlar la no repeticion en cada columna
        if unique_in_column:
            # Para cada columna, se genera una permutacion de los indices (o se usan numeros aleatorios sin repeticion)
            # Siempre que num_tournaments <= pop_size.
            if num_tournaments > pop_size:
                # Si se requieren mas torneos que individuos, se hace sin la restriccion por columna.
                tournament_matrix = np.random.randint(0, pop_size, size=(num_tournaments, tournament_size))
            else:
                cols = []
                for j in range(tournament_size):
                    # Para la columna j, se toman num_tournaments indices sin repeticion
                    perm = np.random.permutation(pop_size)
                    cols.append(perm[:num_tournaments])
                tournament_matrix = np.column_stack(cols)
        else:
            # Sin restricciones, se muestrea con reemplazo para cada candidato.
            tournament_matrix = np.random.randint(0, pop_size, size=(num_tournaments, tournament_size))
    
    # Para cada torneo (fila de la matriz), se selecciona el candidato con el menor fitness.
    winners = []
    for row in tournament_matrix:
        row_fitness = fitness[row]
        winner_index = row[np.argmin(row_fitness)]
        winners.append(winner_index)
    winners = np.array(winners)
    return winners, tournament_matrix
\end{lstlisting}

\section{Archivo \texttt{crossover.py}}
\begin{lstlisting}[
  caption={Implementación de \texttt{crossover.py}},
  style=pythonstyle,
  basicstyle=\ttfamily\footnotesize
]
import numpy as np

def sbx_crossover(parent1, parent2, lower_bound, upper_bound, eta, crossover_prob):
    """Realiza el cruzamiento SBX para dos padres y devuelve dos hijos."""
    child1 = np.empty_like(parent1)
    child2 = np.empty_like(parent2)
    
    if np.random.rand() <= crossover_prob:
        for i in range(len(parent1)):
            u = np.random.rand()
            if u <= 0.5:
                beta = (2*u)**(1/(eta+1))
            else:
                beta = (1/(2*(1-u)))**(1/(eta+1))
            
            # Genera los dos hijos
            child1[i] = 0.5*((1+beta)*parent1[i] + (1-beta)*parent2[i])
            child2[i] = 0.5*((1-beta)*parent1[i] + (1+beta)*parent2[i])
            
            # Asegurar que los hijos esten dentro de los limites
            child1[i] = np.clip(child1[i], lower_bound[i], upper_bound[i])
            child2[i] = np.clip(child2[i], lower_bound[i], upper_bound[i])
    else:
        child1 = parent1.copy()
        child2 = parent2.copy()
    
    return child1, child2

def sbx_crossover_with_boundaries(parent1, parent2, lower_bound, upper_bound,
                                  eta, crossover_prob, use_global_u=False, global_u=None):
    """
    Realiza el cruzamiento SBX con limites, usando formulas que ajustan beta en funcion
    de la cercania a las fronteras. Permite usar un unico 'u' global para todos los individuos 
    de la generacion o, de forma estandar, un 'u' distinto por cada gen.
    
    Args:
      - parent1, parent2: arrays con los padres.
      - lower_bound, upper_bound: arrays con los limites inferiores y superiores.
      - eta: indice de distribucion para SBX.
      - crossover_prob: probabilidad de aplicar el cruce.
      - use_global_u: si es True se utilizara el mismo valor de 'u' para todas las variables.
      - global_u: valor de 'u' que se aplicara globalmente (si se proporciona).
      
    Returns:
      - child1, child2: arrays con los hijos resultantes.
    """
    parent1 = np.asarray(parent1)
    parent2 = np.asarray(parent2)
    child1 = np.empty_like(parent1)
    child2 = np.empty_like(parent2)
    
    # Si no se realiza el crossover, retornamos copias de los padres.
    if np.random.rand() > crossover_prob:
        return parent1.copy(), parent2.copy()
    
    # Si se quiere usar un 'u' global y no se ha pasado, se genera uno.
    if use_global_u:
        if global_u is None:
            global_u = np.random.rand()
    
    for i in range(len(parent1)):
        x1 = parent1[i]
        x2 = parent2[i]
        lb = lower_bound[i]
        ub = upper_bound[i]
        
        # Aseguramos que x1 sea menor o igual que x2
        if x1 > x2:
            x1, x2 = x2, x1
        
        dist = x2 - x1
        if dist < 1e-14:
            child1[i] = x1
            child2[i] = x2
            continue
        
        # Calcular la minima distancia a las fronteras
        min_val = min(x1 - lb, ub - x2)
        if min_val < 0:
            min_val = 0
        
        beta = 1.0 + (2.0 * min_val / dist)
        alpha = 2.0 - beta**(-(eta+1))
        
        # Si se usa u global, se usa el mismo valor para cada variable
        if use_global_u:
            u = global_u
        else:
            u = np.random.rand()
        
        if u <= (1.0 / alpha):
            betaq = (alpha * u)**(1.0/(eta+1))
        else:
            betaq = (1.0 / (2.0 - alpha*u))**(1.0/(eta+1))
        
        # Calcular los hijos
        c1 = 0.5 * ((x1 + x2) - betaq * (x2 - x1))
        c2 = 0.5 * ((x1 + x2) + betaq * (x2 - x1))
        
        # Ajustar a los limites
        child1[i] = np.clip(c1, lb, ub)
        child2[i] = np.clip(c2, lb, ub)
    
    return child1, child2
\end{lstlisting}

\section{Archivo \texttt{mutation.py}}
\begin{lstlisting}[
  caption={Implementación de \texttt{mutation.py}},
  style=pythonstyle,
  basicstyle=\ttfamily\footnotesize
]
import numpy as np

def polynomial_mutation(child, lower_bound, upper_bound, mutation_prob, eta_mut):
    """Aplica mutacion polinomial a un hijo."""
    mutant = child.copy()
    for i in range(len(child)):
        if np.random.rand() < mutation_prob:
            r = np.random.rand()
            diff = upper_bound[i] - lower_bound[i]
            if r < 0.5:
                delta = (2*r)**(1/(eta_mut+1)) - 1
            else:
                delta = 1 - (2*(1-r))**(1/(eta_mut+1))
            mutant[i] = child[i] + delta * diff
            mutant[i] = np.clip(mutant[i], lower_bound[i], upper_bound[i])
    return mutant


def polynomial_mutation_with_boundaries(child, lower_bound, upper_bound,
                                        mutation_prob, eta_mut,
                                        use_global_r=False, global_r=None):
    """
    Aplica mutacion polinomial (con limites) a un vector 'child'.
    Puede usar un unico 'r' global para todas las variables (si use_global_r=True)
    o generar un 'r' distinto para cada variable.
    
    Args:
      - child : array-like
          Cromosoma (vector de decision) a mutar.
      - lower_bound, upper_bound : array-like
          Limites inferiores y superiores para cada variable.
      - mutation_prob : float
          Probabilidad de mutacion (en [0,1]) para cada variable.
      - eta_mut : float
          indice de distribucion para la mutacion.
      - use_global_r : bool
          Si True, se utiliza un unico valor 'r' para todas las variables.
      - global_r : float, opcional
          Valor de 'r' global a usar; si no se proporciona, se genera uno.
    
    Retutrns:
      - mutant : np.ndarray
          Nuevo vector mutado (manteniendo la dimension de 'child').
    """
    mutant = np.array(child, copy=True, dtype=float)
    num_vars = len(child)
    
    # Si se desea usar un 'r' global y no se ha proporcionado, se genera uno una sola vez.
    if use_global_r:
        if global_r is None:
            global_r = np.random.rand()
    
    for i in range(num_vars):
        # Decidir si mutar esta variable
        if np.random.rand() < mutation_prob:
            x = mutant[i]
            xl = lower_bound[i]
            xu = upper_bound[i]
            
            # Evitar division por cero si los limites son casi iguales
            if abs(xu - xl) < 1e-14:
                continue

            # d = distancia normalizada al limite mas cercano
            d = min(xu - x, x - xl) / (xu - xl)
            
            # Elegir r: global o individual para cada variable
            if use_global_r:
                r = global_r
            else:
                r = np.random.rand()

            nm = eta_mut + 1.0

            # Calcular delta_q segun el valor de r
            if r < 0.5:
                bl = 2.0 * r + (1.0 - 2.0 * r) * ((1.0 - d) ** nm)
                delta_q = (bl ** (1.0 / nm)) - 1.0
            else:
                bl = 2.0 * (1.0 - r) + 2.0 * (r - 0.5) * ((1.0 - d) ** nm)
                delta_q = 1.0 - (bl ** (1.0 / nm))
            
            # Calcular la nueva posicion y asegurarse que este dentro de los limites
            y = x + delta_q * (xu - xl)
            mutant[i] = np.clip(y, xl, xu)
    
    return mutant
\end{lstlisting}

\section{Archivo \texttt{auxiliares\_functions.py}}
\begin{lstlisting}[
  caption={Implementación de \texttt{auxiliares\_functions.py}},
  style=pythonstyle,
  basicstyle=\ttfamily\footnotesize
]
import numpy as np
# ---------------------------
# Funciones auxiliares del GA
# ---------------------------
def initialize_population(pop_size, num_variables, lower_bound, upper_bound):
    """Inicializa la poblacion uniformemente en el espacio de busqueda."""
    return np.random.uniform(low=lower_bound, high=upper_bound, size=(pop_size, num_variables))
\end{lstlisting}
\chapter{Registro de indicadores completo.}

\section{Langermann}

\subsection{Resumenes}
\begin{longtable}{lrrrrrrr}
\caption{Archivo: resumen\_run\_1}\label{tab:resumen_run_1} \\
\toprule
Indicador & x0 & x1 & x2 & x3 & x4 & x5 & Fitness \\
\midrule
\endfirsthead
\toprule
Indicador & x0 & x1 & x2 & x3 & x4 & x5 & Fitness \\
\midrule
\endhead
\midrule
\multicolumn{8}{r}{Continued on next page} \\
\midrule
\endfoot
\bottomrule
\endlastfoot
Mejor & 0.1631 & 0.258 & 0.0028 & 0.2868 & 0.0393 & 0.25 & 0.0042 \\
Media & 0.1679 & 0.2601 & 0.0028 & 0.2883 & 0.0403 & 0.2539 & 0.0936 \\
Peor & 0.2285 & 0.258 & 0.0028 & 0.2868 & 0.0393 & 0.3102 & 1.5809 \\
\end{longtable}
\begin{longtable}{lrrrrrrr}
\caption{Archivo: resumen\_run\_2}\label{tab:resumen_run_2} \\
\toprule
Indicador & x0 & x1 & x2 & x3 & x4 & x5 & Fitness \\
\midrule
\endfirsthead
\toprule
Indicador & x0 & x1 & x2 & x3 & x4 & x5 & Fitness \\
\midrule
\endhead
\midrule
\multicolumn{8}{r}{Continued on next page} \\
\midrule
\endfoot
\bottomrule
\endlastfoot
Mejor & 0.0 & 0.0986 & 0.4 & 0.0915 & 0.4 & 0.0116 & -0.6744 \\
Media & 0.0 & 0.1015 & 0.4 & 0.0939 & 0.4 & 0.0118 & -0.6422 \\
Peor & 0.0 & 0.1615 & 0.4 & 0.0915 & 0.4 & 0.019 & -0.1839 \\
\end{longtable}
\begin{longtable}{lrrr}
\caption{Archivo: resumen\_run\_3}\label{tab:resumen_run_3} \\
\toprule
Indicador & x1 & x2 & Fitness \\
\midrule
\endfirsthead
\toprule
Indicador & x1 & x2 & Fitness \\
\midrule
\endhead
\midrule
\multicolumn{4}{r}{Continued on next page} \\
\midrule
\endfoot
\bottomrule
\endlastfoot
Mejor & 2.0167622431775043 & 0.9775688923221356 & -5.14581328834533 \\
Media & 0.8578923228636298 & 0.1119520311959549 & -2.034501886600382 \\
Peor & 1.158397366850342 & 0.1061677488445461 & -0.0597089267754305 \\
Desv. estándar & nan & nan & 0.5028469837532046 \\
\end{longtable}
\begin{longtable}{lrrrrrrr}
\caption{Archivo: resumen\_run\_4}\label{tab:resumen_run_4} \\
\toprule
Indicador & x0 & x1 & x2 & x3 & x4 & x5 & Fitness \\
\midrule
\endfirsthead
\toprule
Indicador & x0 & x1 & x2 & x3 & x4 & x5 & Fitness \\
\midrule
\endhead
\midrule
\multicolumn{8}{r}{Continued on next page} \\
\midrule
\endfoot
\bottomrule
\endlastfoot
Mejor & 0.2503 & 0.2307 & 0.0016 & 0.2188 & 0.1193 & 0.1792 & 0.0022 \\
Media & 0.2525 & 0.2335 & 0.0016 & 0.2217 & 0.1213 & 0.182 & 0.0807 \\
Peor & 0.2959 & 0.2822 & 0.0016 & 0.2738 & 0.1193 & 0.1792 & 2.3144 \\
\end{longtable}
\begin{longtable}{lrrr}
\caption{Archivo: resumen\_run\_5}\label{tab:resumen_run_5} \\
\toprule
Indicador & x1 & x2 & Fitness \\
\midrule
\endfirsthead
\toprule
Indicador & x1 & x2 & Fitness \\
\midrule
\endhead
\midrule
\multicolumn{4}{r}{Continued on next page} \\
\midrule
\endfoot
\bottomrule
\endlastfoot
Mejor & 1.998494126259732 & 1.011705715657761 & -5.161210706524714 \\
Media & 0.9984353322108398 & 0.488488465533086 & -1.1010933566084198 \\
Peor & 2.6256859752208976 & 1.5033712732171227 & 1.604540573899825 \\
Desv. estándar & nan & nan & 1.0800232599648292 \\
\end{longtable}
\begin{longtable}{lrrr}
\caption{Archivo: resumen\_run\_6}\label{tab:resumen_run_6} \\
\toprule
Indicador & x1 & x2 & Fitness \\
\midrule
\endfirsthead
\toprule
Indicador & x1 & x2 & Fitness \\
\midrule
\endhead
\midrule
\multicolumn{4}{r}{Continued on next page} \\
\midrule
\endfoot
\bottomrule
\endlastfoot
Mejor & -0.0120281638507773 & 0.0119344030286503 & -0.9896286873304896 \\
Media & 1.1078639459141058 & 1.1319204270442176 & -0.5447942472082268 \\
Peor & 1.1458796638728408 & 1.169938981286806 & -0.5074559004809934 \\
Desv. estándar & nan & nan & 0.0622931458098346 \\
\end{longtable}
\begin{longtable}{lrrr}
\caption{Archivo: resumen\_run\_7}\label{tab:resumen_run_7} \\
\toprule
Indicador & x1 & x2 & Fitness \\
\midrule
\endfirsthead
\toprule
Indicador & x1 & x2 & Fitness \\
\midrule
\endhead
\midrule
\multicolumn{4}{r}{Continued on next page} \\
\midrule
\endfoot
\bottomrule
\endlastfoot
Mejor & 2.0019436362996443 & 1.0037506288328364 & -5.162016654905215 \\
Media & 0.1192414512066273 & 0.0353625964394437 & 0.6120035661402777 \\
Peor & 0.0252458179930767 & 0.003505827249705 & 1.0031083996798786 \\
Desv. estándar & nan & nan & 0.9666580436507198 \\
\end{longtable}
\begin{longtable}{lrrr}
\caption{Archivo: resumen\_run\_8}\label{tab:resumen_run_8} \\
\toprule
Indicador & x1 & x2 & Fitness \\
\midrule
\endfirsthead
\toprule
Indicador & x1 & x2 & Fitness \\
\midrule
\endhead
\midrule
\multicolumn{4}{r}{Continued on next page} \\
\midrule
\endfoot
\bottomrule
\endlastfoot
Mejor & 0.0153481319147759 & -0.0152376917924069 & -0.9831402532063 \\
Media & -1.906644597007218 & -1.9384580477844384 & -0.2019478608654046 \\
Peor & -1.9878786813725344 & -2.019433856241915 & -0.0248051240544462 \\
Desv. estándar & nan & nan & 0.1303563220305934 \\
\end{longtable}
\begin{longtable}{lrrr}
\caption{Archivo: resumen\_run\_9}\label{tab:resumen_run_9} \\
\toprule
Indicador & x1 & x2 & Fitness \\
\midrule
\endfirsthead
\toprule
Indicador & x1 & x2 & Fitness \\
\midrule
\endhead
\midrule
\multicolumn{4}{r}{Continued on next page} \\
\midrule
\endfoot
\bottomrule
\endlastfoot
Mejor & 2.002592498731848 & 1.006380979616417 & -5.162121824564412 \\
Media & 2.1147866299358484 & 1.1024062009290632 & -4.895316803411505 \\
Peor & 2.147627653713365 & 1.132626689731959 & -4.804433929172063 \\
Desv. estándar & nan & nan & 0.0509847493971701 \\
\end{longtable}
\begin{longtable}{lrrr}
\caption{Archivo: resumen\_run\_10}\label{tab:resumen_run_10} \\
\toprule
Indicador & x1 & x2 & Fitness \\
\midrule
\endfirsthead
\toprule
Indicador & x1 & x2 & Fitness \\
\midrule
\endhead
\midrule
\multicolumn{4}{r}{Continued on next page} \\
\midrule
\endfoot
\bottomrule
\endlastfoot
Mejor & 2.0058434821635696 & 1.0004095459845868 & -5.16143210203958 \\
Media & 0.8724703163431755 & 0.1101092286402504 & -2.377918278148532 \\
Peor & 1.3775297218210725 & 0.4582047123613642 & 2.187761548577683 \\
Desv. estándar & nan & nan & 0.4930068965724287 \\
\end{longtable}

\section{Drop Wave}

\subsection{Resumenes}
\begin{longtable}{lrrrrrrr}
\caption{Archivo: resumen\_run\_1}\label{tab:resumen_run_1} \\
\toprule
Indicador & x0 & x1 & x2 & x3 & x4 & x5 & Fitness \\
\midrule
\endfirsthead
\toprule
Indicador & x0 & x1 & x2 & x3 & x4 & x5 & Fitness \\
\midrule
\endhead
\midrule
\multicolumn{8}{r}{Continued on next page} \\
\midrule
\endfoot
\bottomrule
\endlastfoot
Mejor & 0.1631 & 0.258 & 0.0028 & 0.2868 & 0.0393 & 0.25 & 0.0042 \\
Media & 0.1679 & 0.2601 & 0.0028 & 0.2883 & 0.0403 & 0.2539 & 0.0936 \\
Peor & 0.2285 & 0.258 & 0.0028 & 0.2868 & 0.0393 & 0.3102 & 1.5809 \\
\end{longtable}
\begin{longtable}{lrrrrrrr}
\caption{Archivo: resumen\_run\_2}\label{tab:resumen_run_2} \\
\toprule
Indicador & x0 & x1 & x2 & x3 & x4 & x5 & Fitness \\
\midrule
\endfirsthead
\toprule
Indicador & x0 & x1 & x2 & x3 & x4 & x5 & Fitness \\
\midrule
\endhead
\midrule
\multicolumn{8}{r}{Continued on next page} \\
\midrule
\endfoot
\bottomrule
\endlastfoot
Mejor & 0.0 & 0.0986 & 0.4 & 0.0915 & 0.4 & 0.0116 & -0.6744 \\
Media & 0.0 & 0.1015 & 0.4 & 0.0939 & 0.4 & 0.0118 & -0.6422 \\
Peor & 0.0 & 0.1615 & 0.4 & 0.0915 & 0.4 & 0.019 & -0.1839 \\
\end{longtable}
\begin{longtable}{lrrr}
\caption{Archivo: resumen\_run\_3}\label{tab:resumen_run_3} \\
\toprule
Indicador & x1 & x2 & Fitness \\
\midrule
\endfirsthead
\toprule
Indicador & x1 & x2 & Fitness \\
\midrule
\endhead
\midrule
\multicolumn{4}{r}{Continued on next page} \\
\midrule
\endfoot
\bottomrule
\endlastfoot
Mejor & 2.0167622431775043 & 0.9775688923221356 & -5.14581328834533 \\
Media & 0.8578923228636298 & 0.1119520311959549 & -2.034501886600382 \\
Peor & 1.158397366850342 & 0.1061677488445461 & -0.0597089267754305 \\
Desv. estándar & nan & nan & 0.5028469837532046 \\
\end{longtable}
\begin{longtable}{lrrrrrrr}
\caption{Archivo: resumen\_run\_4}\label{tab:resumen_run_4} \\
\toprule
Indicador & x0 & x1 & x2 & x3 & x4 & x5 & Fitness \\
\midrule
\endfirsthead
\toprule
Indicador & x0 & x1 & x2 & x3 & x4 & x5 & Fitness \\
\midrule
\endhead
\midrule
\multicolumn{8}{r}{Continued on next page} \\
\midrule
\endfoot
\bottomrule
\endlastfoot
Mejor & 0.2503 & 0.2307 & 0.0016 & 0.2188 & 0.1193 & 0.1792 & 0.0022 \\
Media & 0.2525 & 0.2335 & 0.0016 & 0.2217 & 0.1213 & 0.182 & 0.0807 \\
Peor & 0.2959 & 0.2822 & 0.0016 & 0.2738 & 0.1193 & 0.1792 & 2.3144 \\
\end{longtable}
\begin{longtable}{lrrr}
\caption{Archivo: resumen\_run\_5}\label{tab:resumen_run_5} \\
\toprule
Indicador & x1 & x2 & Fitness \\
\midrule
\endfirsthead
\toprule
Indicador & x1 & x2 & Fitness \\
\midrule
\endhead
\midrule
\multicolumn{4}{r}{Continued on next page} \\
\midrule
\endfoot
\bottomrule
\endlastfoot
Mejor & 1.998494126259732 & 1.011705715657761 & -5.161210706524714 \\
Media & 0.9984353322108398 & 0.488488465533086 & -1.1010933566084198 \\
Peor & 2.6256859752208976 & 1.5033712732171227 & 1.604540573899825 \\
Desv. estándar & nan & nan & 1.0800232599648292 \\
\end{longtable}
\begin{longtable}{lrrr}
\caption{Archivo: resumen\_run\_6}\label{tab:resumen_run_6} \\
\toprule
Indicador & x1 & x2 & Fitness \\
\midrule
\endfirsthead
\toprule
Indicador & x1 & x2 & Fitness \\
\midrule
\endhead
\midrule
\multicolumn{4}{r}{Continued on next page} \\
\midrule
\endfoot
\bottomrule
\endlastfoot
Mejor & -0.0120281638507773 & 0.0119344030286503 & -0.9896286873304896 \\
Media & 1.1078639459141058 & 1.1319204270442176 & -0.5447942472082268 \\
Peor & 1.1458796638728408 & 1.169938981286806 & -0.5074559004809934 \\
Desv. estándar & nan & nan & 0.0622931458098346 \\
\end{longtable}
\begin{longtable}{lrrr}
\caption{Archivo: resumen\_run\_7}\label{tab:resumen_run_7} \\
\toprule
Indicador & x1 & x2 & Fitness \\
\midrule
\endfirsthead
\toprule
Indicador & x1 & x2 & Fitness \\
\midrule
\endhead
\midrule
\multicolumn{4}{r}{Continued on next page} \\
\midrule
\endfoot
\bottomrule
\endlastfoot
Mejor & 2.0019436362996443 & 1.0037506288328364 & -5.162016654905215 \\
Media & 0.1192414512066273 & 0.0353625964394437 & 0.6120035661402777 \\
Peor & 0.0252458179930767 & 0.003505827249705 & 1.0031083996798786 \\
Desv. estándar & nan & nan & 0.9666580436507198 \\
\end{longtable}
\begin{longtable}{lrrr}
\caption{Archivo: resumen\_run\_8}\label{tab:resumen_run_8} \\
\toprule
Indicador & x1 & x2 & Fitness \\
\midrule
\endfirsthead
\toprule
Indicador & x1 & x2 & Fitness \\
\midrule
\endhead
\midrule
\multicolumn{4}{r}{Continued on next page} \\
\midrule
\endfoot
\bottomrule
\endlastfoot
Mejor & 0.0153481319147759 & -0.0152376917924069 & -0.9831402532063 \\
Media & -1.906644597007218 & -1.9384580477844384 & -0.2019478608654046 \\
Peor & -1.9878786813725344 & -2.019433856241915 & -0.0248051240544462 \\
Desv. estándar & nan & nan & 0.1303563220305934 \\
\end{longtable}
\begin{longtable}{lrrr}
\caption{Archivo: resumen\_run\_9}\label{tab:resumen_run_9} \\
\toprule
Indicador & x1 & x2 & Fitness \\
\midrule
\endfirsthead
\toprule
Indicador & x1 & x2 & Fitness \\
\midrule
\endhead
\midrule
\multicolumn{4}{r}{Continued on next page} \\
\midrule
\endfoot
\bottomrule
\endlastfoot
Mejor & 2.002592498731848 & 1.006380979616417 & -5.162121824564412 \\
Media & 2.1147866299358484 & 1.1024062009290632 & -4.895316803411505 \\
Peor & 2.147627653713365 & 1.132626689731959 & -4.804433929172063 \\
Desv. estándar & nan & nan & 0.0509847493971701 \\
\end{longtable}
\begin{longtable}{lrrr}
\caption{Archivo: resumen\_run\_10}\label{tab:resumen_run_10} \\
\toprule
Indicador & x1 & x2 & Fitness \\
\midrule
\endfirsthead
\toprule
Indicador & x1 & x2 & Fitness \\
\midrule
\endhead
\midrule
\multicolumn{4}{r}{Continued on next page} \\
\midrule
\endfoot
\bottomrule
\endlastfoot
Mejor & 2.0058434821635696 & 1.0004095459845868 & -5.16143210203958 \\
Media & 0.8724703163431755 & 0.1101092286402504 & -2.377918278148532 \\
Peor & 1.3775297218210725 & 0.4582047123613642 & 2.187761548577683 \\
Desv. estándar & nan & nan & 0.4930068965724287 \\
\end{longtable}

\end{document}


\end{document}
